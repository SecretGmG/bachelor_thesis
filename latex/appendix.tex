\appendix
\chapter{Algebra}
\section{Rewriting the LTD expression}\label{sec:improved_ltd}
We now want to rewrite this expression into a form that is better suited for numerical integration.
For this we will make heavy use of the following identities.
\begin{equation}
    \frac{1}{xy} = \frac{1}{x-y}\qty(\frac{1}{y}-\frac{1}{x})
\end{equation}
The following relations also hold for the $\eta$ coefficients.
\begin{align}
    \HYPER{i}{j} &= -\HYPER{j}{i}               \label{eq:hyper_antisym}\\
    \ELIPT{i}{j}-\HYPER{i}{j} &= 2E_j           \label{eq:elipt_min_hyper}\\
    \ELIPT{i}{k}-\ELIPT{j}{k} &= \HYPER{i}{j}   \label{eq:elipt_min_elipt}\\
    \HYPER{i}{j}+\HYPER{j}{k}+\HYPER{k}{i} &= 0 \label{eq:hyper_cycle}
\end{align}
To start we can notice the following useful identity
\begin{align}
    \frac{1}{\ELIPT{i}{j}\HYPER{i}{j}} &= \frac{1}{2E_j}\qty(\frac{1}{\HYPER{i}{j}} - \frac{1}{\ELIPT{i}{j}}) \label{eq:rewrite_1}
\end{align}
Applying Eq.~\eqref{eq:rewrite_1} to the integrand we get
%\begin{align}
%I(q,p,m_\psi) =& \int \frac{\dd[3]{k}}{{(2\pi)}^3} \frac{1}{(2E_1)(2E_2)(2E_3)} \\
%&  \sum_{i=1}^{3} \prod_{i\neq j}\left(\frac{1}{\HYPER{i}{j}}-\frac{i}{\ELIPT{i}{j}}\right)
%\end{align}
\newcommand{\HMFRAC}[2]{\left(\frac{1}{\HYPER{#1}{#2}}-\frac{1}{\ELIPT{#1}{#2}}\right)}
\begin{align*}
I(q,p,m_\psi) =&  \int \frac{\dd[3]{k}}{{(2\pi)}^3} \frac{1}{(2E_1)(2E_2)(2E_3)} \Bigg[\\
      &  \phantom{{}+{}}
          \left(\HMFRAC{1}{2}\right)\left(\HMFRAC{1}{3}\right)\\
      &  +\left(\HMFRAC{2}{1}\right)\left(\HMFRAC{2}{3}\right)\\
      &  +\left(\HMFRAC{3}{1}\right)\left(\HMFRAC{3}{2}\right)
      \Bigg]
\end{align*}
This can be expanded
\begin{align*}
I(q,p,m_\psi)
&= \int \int \frac{\dd[3]{k}}{{(2\pi)}^3} \frac{1}{(2E_1)(2E_2)(2E_3)} \Bigg[\\
&  \frac{1}{\HYPER{1}{2}\HYPER{1}{3}}
 - \frac{1}{\HYPER{1}{2}\ELIPT{1}{3}}
 - \frac{1}{\ELIPT{1}{2}\HYPER{1}{3}}
 + \frac{1}{\ELIPT{1}{2}\ELIPT{1}{3}}\\
&+ \frac{1}{\HYPER{2}{1}\HYPER{2}{3}}
 - \frac{1}{\HYPER{2}{1}\ELIPT{2}{3}}
 - \frac{1}{\ELIPT{2}{1}\HYPER{2}{3}}
 + \frac{1}{\ELIPT{2}{1}\ELIPT{2}{3}}\\
&+ \frac{1}{\HYPER{3}{1}\HYPER{3}{2}}
 - \frac{1}{\HYPER{3}{1}\ELIPT{3}{2}}
 - \frac{1}{\ELIPT{3}{1}\HYPER{3}{2}}
 + \frac{1}{\ELIPT{3}{1}\ELIPT{3}{2}}
\Bigg]
\end{align*}
and then rearranged
\begin{align*}
I(q,p,m_\psi)
&=  \int \frac{\dd[3]{k}}{{(2\pi)}^3} \frac{1}{(2E_1)(2E_2)(2E_3)} \Bigg[\\
&\phantom{{}+{}}
  \frac{1}{\ELIPT{1}{2}\ELIPT{1}{3}}
+ \frac{1}{\ELIPT{2}{1}\ELIPT{2}{3}}
+ \frac{1}{\ELIPT{3}{1}\ELIPT{3}{2}}\\
&
-\frac{1}{\HYPER{1}{2}}\qty(\frac{1}{\ELIPT{1}{3}} - \frac{1}{\ELIPT{2}{3}})
-\frac{1}{\HYPER{3}{1}}\qty(\frac{1}{\ELIPT{3}{2}} - \frac{1}{\ELIPT{1}{2}})
-\frac{1}{\HYPER{2}{3}}\qty(\frac{1}{\ELIPT{2}{1}} - \frac{1}{\ELIPT{3}{1}})\\
&
+ \frac{1}{\HYPER{1}{2}\HYPER{1}{3}}
+ \frac{1}{\HYPER{2}{1}\HYPER{2}{3}}
+ \frac{1}{\HYPER{3}{1}\HYPER{3}{2}}
\Bigg]
\end{align*}
%We will now show that the purely hyperbolic terms vanish, for this we need
%\begin{align}
%    \frac{1}{\HYPER{i}{j}\HYPER{i}{k}} &= \frac{1}{\HYPER{j}{k}}\qty(\frac{1}{\HYPER{i}{j}} - \frac{1}{\HYPER{i}{k}})\\
%    &= \frac{\HYPER{i}{k}-\HYPER{i}{j}}{\HYPER{i}{j}\HYPER{i}{j}\HYPER{i}{k}} \label{eq:hyper_comm_denom}
%\end{align}
%If we apply Eq.~\eqref{eq:hyper_comm_denom} to each term we get
%\begin{align}
%    \frac{-\HYPER{1}{2}-\HYPER{1}{3}-\HYPER{2}{1}-\HYPER{2}{3}-\HYPER{3}{1}-\HYPER{3}{2}}{\HYPER{1}{2}\HYPER{2}{3}\HYPER{3}{1}}
%    &=
%    \frac{\HYPER{1}{2}-\HYPER{1}{2}+\HYPER{3}{1}-\HYPER{1}{3}+\HYPER{2}{3}-\HYPER{3}{2}}{\HYPER{1}{2}\HYPER{2}{3}\HYPER{3}{1}} = 0
%\end{align}
We will now show that the purely hyperbolic terms vanish,
for this we can simply put the on a common denominator and apply Eq.~\eqref{eq:hyper_cycle}.
\begin{equation}
    \frac{\HYPER{1}{2}+\HYPER{2}{3}+\HYPER{3}{1}}{\HYPER{1}{3}\HYPER{2}{3}\HYPER{3}{1}} = 0
\end{equation}
We now only need to take care of the mixed terms. For this we can use the identity
\begin{align}
    -\frac{1}{\HYPER{i}{j}} \qty(\frac{1}{\ELIPT{i}{k}}-\frac{1}{\ELIPT{j}{k}})
&= \frac{1}{\HYPER{i}{j}} \qty(\frac{\ELIPT{i}{k}-\ELIPT{j}{k}}{\ELIPT{i}{k}\ELIPT{j}{k}})\\
&= \frac{1}{\ELIPT{i}{k}\ELIPT{j}{k}} \label{eq:rewrite_2}
\end{align}
Here we used Eq.~\eqref{eq:elipt_min_elipt} in the last step.
Applying Eq.~\eqref{eq:rewrite_2} to each mixed term we get our final result.
\begin{align*}
I(q,p,m_\psi)
&=  \int \frac{\dd[3]{k}}{{(2\pi)}^3} \frac{1}{(2E_1)(2E_2)(2E_3)} \Bigg[\\
  &\frac{1}{\ELIPT{2}{1}\ELIPT{3}{1}}
+ \frac{1}{\ELIPT{1}{2}\ELIPT{1}{3}}
+ \frac{1}{\ELIPT{1}{2}\ELIPT{3}{2}}
+ \frac{1}{\ELIPT{2}{1}\ELIPT{2}{3}}
+ \frac{1}{\ELIPT{1}{3}\ELIPT{2}{3}}
+ \frac{1}{\ELIPT{3}{1}\ELIPT{3}{2}}
\Bigg]
\end{align*}