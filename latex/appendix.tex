\appendix
\section{Rewriting the LTD expression}\label{app:cff}
We now want to rewrite this expression into a form that is better suited for
numerical integration. For this we will make heavy use of the following
identities.
\begin{equation}
    \frac{1}{xy} = \frac{1}{x-y}\qty(\frac{1}{y}-\frac{1}{x})
\end{equation}
The following relations also hold for the $\eta$ coefficients.
\begin{align}
    \HYPER{i}{j}                           & = -\HYPER{j}{i}               \label{eq:hyper_antisym} \\
    \ELIPT{i}{j}-\HYPER{i}{j}              & = 2E_j           \label{eq:elipt_min_hyper}            \\
    \ELIPT{i}{k}-\ELIPT{j}{k}              & = \HYPER{i}{j}   \label{eq:elipt_min_elipt}            \\
    \HYPER{i}{j}+\HYPER{j}{k}+\HYPER{k}{i} & = 0 \label{eq:hyper_cycle}
\end{align}
To start we can notice the following useful identity
\begin{align}
    \frac{1}{\ELIPT{i}{j}\HYPER{i}{j}} & = \frac{1}{2E_j}\qty(\frac{1}{\HYPER{i}{j}} - \frac{1}{\ELIPT{i}{j}}) \label{eq:rewrite_1}
\end{align}
Applying Eq.~\eqref{eq:rewrite_1} to the integrand we get
%\begin{align}
%I(q,p,m_\psi) =& \int \frac{\dd[3]{k}}{{(2\pi)}^3} \frac{1}{(2E_1)(2E_2)(2E_3)} \\
%&  \sum_{i=1}^{3} \prod_{i\neq j}\left(\frac{1}{\HYPER{i}{j}}-\frac{i}{\ELIPT{i}{j}}\right)
%\end{align}
\newcommand{\HMFRAC}[2]{\left(\frac{1}{\HYPER{#1}{#2}}-\frac{1}{\ELIPT{#1}{#2}}\right)}
\begin{align*}
    I(q,p,m_\psi) = & \int \frac{\dd[3]{k}}{{(2\pi)}^3} \frac{1}{(2E_1)(2E_2)(2E_3)} \Bigg[ \\
                    & \phantom{{}+{}}
    \HMFRAC{1}{2}\HMFRAC{1}{3}                                                              \\
                    & +\HMFRAC{2}{1}\HMFRAC{2}{3}                                           \\
                    & +\HMFRAC{3}{1}\HMFRAC{3}{2}
    \Bigg]
\end{align*}
This can be expanded
\begin{align*}
    I(q,p,m_\psi)
     & = \int \frac{\dd[3]{k}}{{(4\pi)}^3} \frac{1}{E_1 E_2 E_3} \Bigg[ \\
     & \frac{1}{\HYPER{1}{2}\HYPER{1}{3}}
        - \frac{1}{\HYPER{1}{2}\ELIPT{1}{3}}
        - \frac{1}{\ELIPT{1}{2}\HYPER{1}{3}}
    + \frac{1}{\ELIPT{1}{2}\ELIPT{1}{3}}                                \\
     & + \frac{1}{\HYPER{2}{1}\HYPER{2}{3}}
        - \frac{1}{\HYPER{2}{1}\ELIPT{2}{3}}
        - \frac{1}{\ELIPT{2}{1}\HYPER{2}{3}}
    + \frac{1}{\ELIPT{2}{1}\ELIPT{2}{3}}                                \\
     & + \frac{1}{\HYPER{3}{1}\HYPER{3}{2}}
        - \frac{1}{\HYPER{3}{1}\ELIPT{3}{2}}
        - \frac{1}{\ELIPT{3}{1}\HYPER{3}{2}}
        + \frac{1}{\ELIPT{3}{1}\ELIPT{3}{2}}
        \Bigg]
\end{align*}
and then rearranged
\begin{align*}
    I(q,p,m_\psi)
     & =  \int \frac{\dd[3]{k}}{{(4\pi)}^3} \frac{1}{E_1E_2E_3} \Bigg[           \\
     & \phantom{{}+{}}
    \frac{1}{\ELIPT{1}{2}\ELIPT{1}{3}}
    + \frac{1}{\ELIPT{2}{1}\ELIPT{2}{3}}
    + \frac{1}{\ELIPT{3}{1}\ELIPT{3}{2}}                                         \\
     &
    -\frac{1}{\HYPER{1}{2}}\qty(\frac{1}{\ELIPT{1}{3}} - \frac{1}{\ELIPT{2}{3}})
    -\frac{1}{\HYPER{3}{1}}\qty(\frac{1}{\ELIPT{3}{2}} - \frac{1}{\ELIPT{1}{2}})
    -\frac{1}{\HYPER{2}{3}}\qty(\frac{1}{\ELIPT{2}{1}} - \frac{1}{\ELIPT{3}{1}}) \\
     &
    + \frac{1}{\HYPER{1}{2}\HYPER{1}{3}}
    + \frac{1}{\HYPER{2}{1}\HYPER{2}{3}}
    + \frac{1}{\HYPER{3}{1}\HYPER{3}{2}}
    \Bigg]
\end{align*}
%We will now show that the purely hyperbolic terms vanish, for this we need
%\begin{align}
%    \frac{1}{\HYPER{i}{j}\HYPER{i}{k}} &= \frac{1}{\HYPER{j}{k}}\qty(\frac{1}{\HYPER{i}{j}} - \frac{1}{\HYPER{i}{k}})\\
%    &= \frac{\HYPER{i}{k}-\HYPER{i}{j}}{\HYPER{i}{j}\HYPER{i}{j}\HYPER{i}{k}} \label{eq:hyper_comm_denom}
%\end{align}
%If we apply Eq.~\eqref{eq:hyper_comm_denom} to each term we get
%\begin{align}
%    \frac{-\HYPER{1}{2}-\HYPER{1}{3}-\HYPER{2}{1}-\HYPER{2}{3}-\HYPER{3}{1}-\HYPER{3}{2}}{\HYPER{1}{2}\HYPER{2}{3}\HYPER{3}{1}}
%    &=
%    \frac{\HYPER{1}{2}-\HYPER{1}{2}+\HYPER{3}{1}-\HYPER{1}{3}+\HYPER{2}{3}-\HYPER{3}{2}}{\HYPER{1}{2}\HYPER{2}{3}\HYPER{3}{1}} = 0
%\end{align}
We will now show that the purely hyperbolic terms vanish, for this we can
simply put the on a common denominator and apply Eq.~\eqref{eq:hyper_cycle}.
\begin{equation}
    \frac{\HYPER{1}{2}+\HYPER{2}{3}+\HYPER{3}{1}}{\HYPER{1}{3}\HYPER{2}{3}\HYPER{3}{1}} = 0
\end{equation}
We now only need to take care of the mixed terms. For this we can use the identity
\begin{align}
    -\frac{1}{\HYPER{i}{j}} \qty(\frac{1}{\ELIPT{i}{k}}-\frac{1}{\ELIPT{j}{k}})
     & = \frac{1}{\HYPER{i}{j}} \qty(\frac{\ELIPT{i}{k}-\ELIPT{j}{k}}{\ELIPT{i}{k}\ELIPT{j}{k}}) \\
     & = \frac{1}{\ELIPT{i}{k}\ELIPT{j}{k}} \label{eq:rewrite_2}
\end{align}
Here we used Eq.~\eqref{eq:elipt_min_elipt} in the last step.
Applying Eq.~\eqref{eq:rewrite_2} to each mixed term we get our final result.
\begin{align*}
    I(q,p,m_\psi)
     & =  \int \frac{\dd[3]{k}}{{(4\pi)}^3} \frac{1}{E_1E_2E_3} \Bigg[ \\
     & \frac{1}{\ELIPT{2}{1}\ELIPT{3}{1}}
        + \frac{1}{\ELIPT{1}{2}\ELIPT{1}{3}}
        + \frac{1}{\ELIPT{1}{2}\ELIPT{3}{2}}
        + \frac{1}{\ELIPT{2}{1}\ELIPT{2}{3}}
        + \frac{1}{\ELIPT{1}{3}\ELIPT{2}{3}}
        + \frac{1}{\ELIPT{3}{1}\ELIPT{3}{2}}
        \Bigg]
\end{align*}

\section{OneLOopBridge}\label{app:oneloopbridge}

\textit{OneLOopBridge} provides a lightweight Rust and Python interface to the \textit{OneLOop} Fortran library, providing access to scalar one-loop integrals with up to four external legs. The package was developed to offer a safe, idiomatic, and reproducible way to evaluate loop integrals.

The underlying \textit{OneLOop} library implements analytic expressions for
dimensionally regularized scalar one-loop integrals, returning their Laurent
expansion in the dimensional regulator $\varepsilon = (4-d)/2$.
\textit{OneLOopBridge} exposes these routines by providing save Rust bindings
and a Python interface. In addition to the computation of integrals, and
interface is provided to access the most important internal settings of
\textit{OneLOop}, namely:
\begin{itemize}
    \item the renormalization scale,
    \item the on-shell threshold,
    \item and the internal logging level.
\end{itemize}

All integral routines return a \texttt{ResultOLO} object containing the
coefficients of the Laurent expansion,
\begin{equation}
    I = \varepsilon^{-2} I_{-2} + \varepsilon^{-1} I_{-1} + I_{0} + \mathcal{O}(\varepsilon)
\end{equation}
where finite integrals automatically yield vanishing divergent coefficients. The coefficients are accessible individually. A global conversion factor \texttt{TO\_FEYNMAN} is provided to translate results to the conventional Feynman integral normalization.

\subsection{Rust Interface}

The Rust API mirrors the structure of the underlying integrals and emphasizes
explicitness. Each function corresponds to a specific configuration and takes
the relevant kinematic invariants and squared masses as arguments. Complex
masses are supported via \texttt{num\_complex::Complex64}.

Typical usage involves computing Lorentz invariants externally and passing them
to the wrapper functions. The returned \texttt{ResultOLO} struct provides
accessor methods for the Laurent coefficients.

The package can simply be installed using \texttt{cargo}

\begin{minted}{bash}
cargo add --git https://github.com/SecretGmG/OneLOopBridge
\end{minted}

\subsection{Python Interface}
Python bindings are generated using \texttt{maturin} and \texttt{PyO3},
exposing the same functionality as the rust interface. The package includes
python stub files, allowing the users linting and type checking to integrate
the library.

To install and compile the python bindings it is necessary to download the
repository and compile it manually. After that maturin can be run to install the bindings
into an active python environment. This can be done by running the following
inside an active python environment. Note that this will change the current
working directory.
\begin{minted}{bash}
git clone https://github.com/SecretGmG/OneLOopBridge.git
cd OneLOopBridge && make -f Makefile develop
\end{minted}
