\documentclass{article}

%-- My Definitions
\usepackage{mydefs}

%-- Math stuff
\usepackage{mathtools, amsfonts, amsthm, amssymb}
%-- Physics and Chemistry stuff
\usepackage{physics}

\usepackage[version=4]{mhchem}
\usepackage{braket}
\usepackage{siunitx}
\sisetup{
    input-digits = 0123456789\pi,
    separate-uncertainty,
    table-align-uncertainty,
    table-number-alignment = center
}

%to compile Feynman diagrams: uncomment these
%can take a lot of compile time, set the compiler to LuaLaTex
\usepackage[compat=1.1.0]{tikz-feynman}
\usepackage{tikz}

%-- Computer science stuff
\usepackage{algorithm, algpseudocode}
%\usepackage{minted}
%-- Plotting
%\usepackage{graphicx}

%-- Tables and enumerations
\usepackage{enumerate}
\usepackage{booktabs, multirow}
\usepackage{caption, subcaption}

%-- Bibliography
\usepackage{csquotes}
\usepackage[
  backend=biber,
  style=numeric,
  citestyle=numeric,
  sorting=none
]{biblatex}
%\addbibresource{bibliography.bib}

%-- Numbering
\numberwithin{equation}{section}
\numberwithin{algorithm}{section}
\counterwithin{figure}{section}
\counterwithin{table}{section}


%-- General Formatting
\usepackage[top=20mm,bottom=20mm,left=25mm,right=25mm]{geometry}
\usepackage[english]{babel}
\selectlanguage{english}
\usepackage[colorlinks=true, allcolors=blue]{hyperref}
\usepackage{fancyhdr}
\usepackage{parskip}
\usepackage{csquotes}
%\setlength{\parindent}{0pt}
%\setlength{\parskip}{0.5\baselineskip}

%-----> Document begins here <-----

\bibliography{bachelor_thesis.bib}
\begin{document}

% -- Title page

\title{Numerical integration of the scalar Triangle integral with a complex mass using threshold subtraction}
\author{Cedric Sigrist}
\date{\today}
\maketitle

\clearpage

\begin{abstract}
TODO
\end{abstract}
\clearpage

% -- Table of contents
\setcounter{tocdepth}{2}
\tableofcontents

\clearpage

\section{Introduction}
%The numerical evaluation of multi-loop integrals in momentum space requires methods capable of efficiently regulating and isolating singular behavior. Threshold subtraction has proven to be an effective framework in this regard~\cite{kermanschah_numerical_2022} TODO ADD MORE. It was initially developed to address threshold singularities in~\cite{capatti_loop_2019} and subsequently extended to more intricate intersecting structures in~\cite{kermanschah_numerical_2022}.
%
%Existing applications, however, predominantly focus on integrals with real internal masses. In settings involving unstable particles, complex masses are unavoidable, and their presence modifies the location and character of the singularities. Whether the threshold subtraction method remains stable and reliable in this broader context is not a trivial question.
%
%In this work, I investigate the scalar triangle integral with complex internal masses as a controlled environment in which to test this extension. The triangle topology is sufficiently simple to allow for a transparent analysis, yet it retains essential features introduced by complex masses, including shifted branch cuts and modified threshold configurations. Demonstrating that threshold subtraction can be applied successfully in this setting broadens the scope of the method and represents a step toward its use in realistic multi-loop computations formulated within complex-mass schemes.

\section{Scalar Triangle Integral}
I consider the scalar Triangle integral, restricted to complex masses with $\Re(m_i^2) > 0$ and $\Im(m_i^2) \leq 0$.
\begin{figure}[H]
    \centering
     \feynmandiagram[horizontal=t2 to a] {
    % incoming particles on the left
    p1 [x=0, y=0]  -- [momentum =\(\vb{p}_1\)] t1 [x=3, y=0] , 
    p2 [x=0, y=-4] -- [momentum =\(\vb{p}_2\)] t3 [x=3, y=-4],

    % triangle loop
    t1 -- [momentum =\(\vb{k} + \vb{p}_1\), edge label' = \(m_1\)] t2 
    -- [momentum =\(\vb{k} - \vb{p}_2\), edge label' = \(m_3\)] t3 
    -- [momentum =\(\vb{k}\), edge label' = \(m_2\)] t1,

    % outgoing particle on the right
    t2[x = 6.5, y=-2] -- [momentum=\(\vb{p}_1+\vb{p}_2\)]a [x = 9.5, y = -2],
}; 
   \caption{Feynman diagram of the scalar Triangle integral}\label{fig:triangle_feynman}
\end{figure}
The Feynman diagram in Figure~\ref{fig:triangle_feynman} produces equation~\eqref{eq:triangle_integral}.
\begin{equation}
    I = \int \frac{\dd[4]{k}}{{(2\pi)}^4} \frac{i}{D_1 \, D_2 \, D_3} \label{eq:triangle_integral}
\end{equation}
with
\begin{equation}
    D_i = {(\vb{k}-\vb{q}_i)}^2-{\qty(m_i-i\varepsilon)}^2 = {(\vb{k}-\vb{q}_i)}^2-m_i^2 + i \varepsilon  \label{eq:denom}
\end{equation}
Notice that the feynman causal prescription $-i\varepsilon$ is included in the definition of the denominator. In this thesis, I will sometimes also write the causal prescription implicitly in the masses, i.e. $m_i \to m_i - i\varepsilon$, as it shortens some of the calculations.

The momenta $\vb{q_i}$ are defined as
\begin{align}
    \vb{q}_1 &= \vb{p_1}\\
    \vb{q}_2 &= \vb{0}  \\
    \vb{q}_3 &= -\vb{p_2}
\end{align}

\subsection{Integration of the Loop Energy}
There are many ways to perform the loop energy integration in eq.~\eqref{eq:triangle_integral}. Following the steps in~\cite{catani_loops_2008} to rederive the Loop Tree Duality (LTD) representation by integrating over the energy and then algebraically manipulating the resulting expression to get a result in the Cross Free Family (CFF) representation derived in~\cite{capatti_exposing_2023}.

I will perform the $k^0$ integration with Cauchy's theorem. For this we need to consider the poles of the integrand from Eq.~\eqref{eq:triangle_integral} in $k^0$. These are given by
\begin{equation}
    k^0 = q_i^0 \pm \qty(\sqrt{{\qty(\va{k}-\va{q}_i)}^2+m_i^2} - i\varepsilon) = q_i^0 \pm E_i
\end{equation}
with
\begin{equation}
    E_i = \sqrt{{\qty(\va{k}-\va{q}_i)}^2+m_i^2} - i\varepsilon
\end{equation}
Closing the contour around the poles corresponding to the principal square roots leads to the contour shown in Figure~\ref{fig:poles}
\begin{figure}[H]
    \centering
    \begin{tikzpicture}[scale=1.2,>=latex]

% Axes
\draw[->,gray] (-3.5,0)--(3.5,0) node[right]{$\mathrm{Re}\,k^0$};
\draw[->,gray] (0,-3.5)--(0,1.5) node[above]{$\mathrm{Im}\,k^0$};

% Red poles (values roughly matching Python example)
\foreach \x/\y/\lbl in {
  -2/0.1/{-E1},
  0.5/-0.1/{+E1},
  -1.25/0.1/{-E2},
  1.25/-0.1/{+E2},
  -0.5/0.1/{-E3},
  2/-0.1/{+E3}
}{
  \fill[red] (\x,\y) circle(1.8pt);
  \node[above=3pt] at (\x,\y) {\lbl};
}

% Contour (clockwise semicircle in lower half-plane)
\draw[thick,blue,
  decoration={markings,mark=at position 0.25 with {\arrow{>}},
  mark=at position 0.75 with {\arrow{>}}},
  postaction={decorate}
  ] (3,0) arc[start angle=0,end angle=-180,radius=3];

% Lower line along real axis (closing contour)
\draw[thick,blue,postaction={decorate},
  decoration={markings,mark=at position 0.25 with {\arrow{>}},
  mark=at position 0.75 with {\arrow{>}}}
  ] (-3,0)--(3,0);

% Labels for contour
\node[blue,below right] at (2,-2.2) {$\Gamma$};
\node[below right] at (3,0) {$R$};
\node[below left] at (-3,0) {$-R$};

\end{tikzpicture}
    \caption{Poles of the denominator}\label{fig:poles}
\end{figure}

We can now apply Cauchy's theorem to perform the $k^0$ integral.
\begin{equation}
    I = \int \frac{\dd[3]{k}}{{(2\pi)}^4} 2\pi i \sum_{i= 1}^{3} \On{Res}_{k^0 = q^0_i + E_i}\qty[\frac{i}{D_1D_2D_3}] \label{eq:cauchy}
\end{equation}

Let's compute the residues.
\begin{equation}
    \On{Res}_{k^0 = q^0_i + E_i}\qty[\frac{i}{D_1D_2D_3}] = {\qty[\frac{i}{D_i'}\prod_{i \neq j} \frac{1}{D_j}]}_{k^0=q^0_i + E_i}= \frac{i}{2E_i} \prod_{i\neq j} \eval{\frac{1}{D_j}}_{k^0=q^0_i+E_i} \label{eq:residue}
\end{equation}

\newcommand{\ELIPT}[2]{\eta_{#1#2}^{++}}
\newcommand{\HYPER}[2]{\eta_{#1#2}^{+-}}

We can also introduce $\ELIPT{i}{j}$ called E-Surface and $\HYPER{i}{j}$ called H-Surface.

\begin{align}
    \eval{D_j}_{k^0=E_i} &= {\qty(q^0_i+E_i-q^0_j)}^2 - \overbrace{{\qty(\va{k}-\va{q}_j)}^2+m_j^2}^{E_j^2}\\
    &= E_i^2+{q_i^0}^2+{q_j^0}^2-2q_i^0q_j^0+2q_i^0E_i - 2q_j^0E_i-E_j^2\\
    &= \underbrace{\left(E_i+E_j+q_i^0-q_j^0\right)}_{\ELIPT{i}{j}}\underbrace{\left(E_i-E_j+q_i^0-q_j^0\right)}_{\HYPER{i}{j}} \label{eq:surfaces}
\end{align}
Inserting the results from Eq.~\eqref{eq:surfaces} and Eq.~\eqref{eq:residue} into Eq.~\eqref{eq:cauchy} we get the Loop Tree duality expression.
\newcommand{\BLTD}[3]{2E_{#1}\ELIPT{#1}{#2}\HYPER{#1}{#2}\ELIPT{#1}{#3}\HYPER{#1}{#3}}
\begin{equation}
    I =  \int \frac{\dd[3]{k}}{{(2\pi)}^3}\qty(\frac{1}{\BLTD{1}{2}{3}}+\frac{1}{\BLTD{2}{1}{3}}+\frac{1}{\BLTD{3}{1}{2}}) \label{eq:ltd}
\end{equation}

While this is a nice expression, it is not yet suited for numeric integration, since all of the H-Surfaces contain singularities which would make the procedure numerically unstable. It turns out however that these singularities are spurious and can be removed by purely algebraic manipulations, shown in detail in Appendix~\ref{sec:improved_ltd}. The resulting expression contains only E-Surfaces.

\begin{gather}
    I =  \int \dd[3]{k} \mathcal{I}_{CFF} \\ 
    \mathcal{I}_{CFF} = \frac{1}{{(4\pi)}^3} \frac{1}{E_1E_2E_3} \qty(
    \frac{1}{\ELIPT{2}{1}\ELIPT{3}{1}}
    + \frac{1}{\ELIPT{1}{2}\ELIPT{1}{3}}
    + \frac{1}{\ELIPT{1}{2}\ELIPT{3}{2}}
    + \frac{1}{\ELIPT{2}{1}\ELIPT{2}{3}}
    + \frac{1}{\ELIPT{1}{3}\ELIPT{2}{3}}
    + \frac{1}{\ELIPT{3}{1}\ELIPT{3}{2}}
    )
\label{eq:cff}
\end{gather}

\subsection{Cross Free Family}
Equation~\eqref{eq:cff} can also be derived directly in a more diagrammatic way, as shown in~\cite{capatti_exposing_2023}. TODO

\section{Threshold Subtraction}
Short introduction, maybe with an easy example to follow

\section{E-Surface}
Equation~\eqref{eq:cff} can still contain singularities, which need further treatment to be integrated numerically.
The equation is singular when an E-Surface $\ELIPT{i}{j}$ is zero. For threshold subtraction to work, we will need to characterize these singularities. In the case of complex masses, this characterization needs some extra care. We need to make sure to find the zeros of the E-Surface in the complex plane, rather than in the real plane.

Let us introduce the momentum $\va{k'}$ defined in equation~\eqref{eq:k_prime}.
\begin{equation}
    \va{k'} = \va{k} - \frac{1}{2}(\va{q}_i+\va{q}_j) \label{eq:k_prime}
\end{equation}

Defining $\vb{q} = \frac{1}{2}\qty(\vb{q}_i-\vb{q}_j)$, the E-Surface condition simplifies to the following form.
\begin{equation}
    \ELIPT{i}{j} =  \sqrt{{(\va{k'}-\va{q})}^2+m_i^2} + \sqrt{{(\va{k'}+\va{q})}^2+m_j^2} + 2q^0 \overset{!}{=} 0\label{eq:e_surface}
\end{equation}
\subsubsection{E-Surface existence condition}\label{sec:e_surface_exist}
%Maybe First do simplified case with $m_i = m_j = m$. show that it is indeed ellipsoid.
Noticing that the E-Surface is of the same form as~\eqref{eq:identity} we can perform the following steps to remove the square roots, we need to however carefully track the signs of the terms, to make sure to not introduce extraneous solutions. In practice, we will do this by checking the solutions afterwards, by inserting it into the E-Surface equation and checking if the result equals zero.

%TODO: maybe find GLOBAL existence condition to remove these extraneous solutions

\begin{equation}
    \sqrt{A}+\sqrt{B} + C = 0 \label{eq:identity}
\end{equation}
We can however already find a necessary but not sufficient condition for the existence of the E-Surface. Considering equation~\eqref{eq:identity} and remembering that we assume $\Re(m_i^2) > 0$ and $\Re(m_j^2) > 0$ we find that for any solution to exist, the following identity must hold
\begin{equation}
    C = 2 q^0 < 0
\end{equation}
This condition already excludes exactly half of the E-surfaces from having any solutions. We will now precede with the characterization of the solutions of the identity~\eqref{eq:identity}. Further extraneous solutions, that may be introduced in the following steps will be filtered later by checking. 
\begin{align}
    \sqrt{A}+\sqrt{B} + C = &0\\
    \implies\quad 2\sqrt{AB} &= C^2-A-B\\
    \implies\quad 4AB &= {(C^2-A-B)}^2\\
    \implies\quad C^2 - 2(A+B) + {\qty(\frac{A-B}{C})}^2 &= 0 \label{eq:identity2}
\end{align}
Applying this identity to the E-surface equation we find
\begin{align*}
    A-B &= -4\va{k'}\cdot \va{q} + m_i^2 - m_j^2\\
    A+B &= 2\va{k'}^2 + 2 \va{q}^2 + m_i^2 + m_j^2
\end{align*}
Which yields the following equation after some simplifications.
\begin{equation} 
    4{q^0}^2 - 2 \qty(2\va{k'}^2 + 2 \va{q}^2 + m_i^2 + m_j^2) + {\qty(\frac{-4 \va{k'}\cdot \va{q} + m_i^2 - m_j^2}{-2q^0})}^2= 0
\end{equation}
This equation can be further simplified, showing that it is quadratic in $\va{k}'$. Note, that $\vb{q}^2 = {q^0}^2 - \va{q}^2$.
%\begin{equation}
%    \vb{q}^2-\frac{m_i^2 + m_j^2}{2} - \va{k'}^2 + {\qty(\va{k'}\cdot \frac{\va{q}}{q^0})}^2 - 2 \qty(\frac{m_i^2 - m_j^2}{4q^0}) \qty(\va{k'}\cdot \frac{\va{q}}{q^0}) + {\qty(\frac{m_i^2 - m_j^2}{4q^0})}^2 = 0
%\end{equation}
\begin{equation}
    \va{k'}^2 - {\qty(\va{k'}\cdot\va{v})}^2 - 2 \Delta \qty(\va{k'}\cdot\va{v}) - \vb{q}^2 + \braket{m^2} + \Delta^2 = 0 \label{eq:quad_k_prime}
\end{equation}
with
\begin{equation}
    \va{v} = \frac{\va{q}}{q^0}
\qquad
    \braket{m^2} = \frac{m_i^2 + m_j^2}{2}
\qquad
    \Delta = \frac{m_i^2 - m_j^2}{4q^0}
\end{equation}
To perform threshold subtraction we will need to find the roots of equation~\eqref{eq:quad_k_prime} along a given line $k = k\,\hat{\vb{k}} + \va{k_0}$. In this parametrization we have
\begin{equation}
    \va{k'} =  k\,\hat{\vb{k}} + \underbrace{\va{k_0} - \frac{1}{2}(\va{q}_i+\va{q}_j)}_{\va{k'_0}} \label{eq:parametrization}
\end{equation}
The roots of equation~\eqref{eq:quad_k_prime} in the radial variable $k$ in the parametrization~\eqref{eq:parametrization} are given by the following quadratic equation.

\begin{equation}
    \alpha k^2 + \beta k + \gamma = 0 \label{eq:quadratic_k}
\end{equation}
with
\begin{align}
    \alpha &= 1-{\qty(\hat{\vb{k}}\cdot\va{v})}^2\\
    \beta &= 2(\hat{\vb{k}}\cdot\va{k'_0})-2(\hat{\vb{k}}\cdot\va{v})(\va{k'_0}\cdot\va{v})-2\Delta(\hat{\vb{k}}\cdot\va{v})\\
    \gamma &= \va{k'_0}^2-{\qty(\va{k'_0}\cdot\va{v})}^2-2\Delta(\va{k'_0}\cdot\va{v})-\vb{q}^2+\braket{m^2}+\Delta^2
\end{align}
This quadratic equation can be solved with the standard quadratic formula, and thus always has exactly two complex valued solutions, which we will call $k^*_+$ and $k^*_-$. We now need to check if they actually solve the equation~\eqref{eq:e_surface}, because the derivation of equation~\eqref{eq:identity2} may have introduced extraneous solutions. With this procedure we will either find two solutions, or none. If real valued solutions exist, they will always be found with this procedure.

It is worthwhile to consider the case where $m_i = m_j = m$ to build geometric intuition of the solutions. In this case the quadratic equation~\eqref{eq:quadratic_k} simplifies, and the geometry of the solutions is more transparent. Choosing the parametrization~\eqref{eq:parametrization} with $\va{k'_0} = \va{k_0}$, we get
\begin{equation}
    k^2 = \frac{\vb{q}^2-m^2}{1-{\qty(\hat{\vb{k}} \cdot \va{v})}^2} \label{eq:quadratic_k_simple}
\end{equation}

Considering that if the external momenta are onshell, we find that $\norm{v} < 1$. In the case where $\vb{q}^2-m^2$, where real solutions exist, equation~\eqref{eq:quadratic_k_simple} defines a prolate spheroid.

\section{Threshold subtraction}\label{sec:thresh}
Threshold singularities appear whenever a loop momentum $\va{k}$ is on an E-surface.
These integrable singularities make numerical integration of the integrand impossible, if not treated appropriately

The idea of threshold subtraction is straightforward: isolate the pole generated by a single E-surface and subtract a locally defined counterterm that reproduces its behavior exactly at the singular point. This ensures that the combined integrand remains finite everywhere in the integration domain. In a well chosen parametrization, it suffices to integrate along a a single dimension to integrate over the pole. This integration can be performed analytically, or with some oder stable procedure, thus eliminating the singularity entirely.

In resent approaches, only singularities lying exactly in the integration domain are considered for threshold subtraction~\cite{kermanschah_numerical_2022,kermanschah_numerical_2024,capatti_loop_2019}. In this thesis I extend this idea to singularities that are not exactly in the integration domain, but instead are shifted to the complex plane. While not strictly necessary for the convergence of the integrand, this extension allows for a more general treatment of singularities and improves numerical stability in cases where the singularities are very close to the integration domain.

\subsection{Subtraction of a single E-Surface}
To remove the threshold singularity corresponding to an E-surface, we will define a local counterterm for an integrand of the form
\begin{equation}
    \mathcal{I} = \frac{f(\va{k})}{\ELIPT{i}{j}}
\end{equation}
where the integration space is parametrized hemispherically with the center $\va{k}_0$ chosen inside of the ellipsoid.
This choice is necessary to ensure local cancellations~\cite{}.
The zeros of $\ELIPT{i}{j}$ along the radial direction can then be found with Eq.~\eqref{eq:e_surface_quad}.
There are always two solution, which we will denote $k^*_+$ and $k^*_-$.

The singular behavior of the integrand can fully be captured by the first-order taylor expansion in $\ELIPT{i}{j}$ around the pole. Let us consider only the pole $k^*_{\pm}$ for now.
\begin{equation}
    \ELIPT{i}{j} = 0 + (k-k^*_{\pm}) \underbrace{\eval{\pdv{k} \ELIPT{i}{j}}_{\va{k}=\va{k^*_{\pm}}}}_{\eta'} + \order{k^2}
\end{equation}
We can now define a counterterm that will cancel the singularity in $k^*_{\pm}$.
\begin{equation}
    \on{CT}_{ij}(k^*_{\pm}) = \chi (\va{k}) \frac{f(\va{k^*_{\pm}})}{(k-k^*_{\pm}) \, \eta'}
\end{equation}
where $\chi (\va{k})$ is an arbitrary mask, that is smooth at $\chi (\va{k}^*_{\pm}) = 1$ and vanishing as $\abs{k-k^*_{\pm}} \to \infty$
The partial derivative along the radial variable $k$ is straightforward to compute explicitly.
\begin{equation}
    \eval{\pdv{k} \ELIPT{i}{j}}_{\va{k}=k^*_{\pm} \hat{\vb{k}} } = {\qty[\frac{1}{E_i}(\va{k}-\va{q}_i)\cdot \hat{\vb{k}} + \frac{1}{E_j}(\va{k}-\va{q}_j)\cdot \hat{\vb{k}} ]}_{\va{k} = k^*_{\pm} \hat{\vb{k}}}
\end{equation}

By construction, subtracting the counterterm $\on{CT}_{ij}(k^*_{\pm})$ will remove the singularity at $k^*_{\pm}$ from the integrand. By choosing the mask $\chi$ appropriately we can ensure that it is possible to integrate along the radius of the counterterm analytically. This allows to add the subtracted counterterm back to the final result in an integral in a way that completely removes the singularity.

\subsection{Radially integrating the counterterm}
The choice of the mask $\chi$ determines many properties of the counterterm.
Writing the counterterm integral in hemispherical coordinates centered at $\va{k_0}$, we have

\begin{equation}
    I_{\on{CT}} = \frac{1}{2} \int_{S^2} \dd[2]{\hat{\vb{k}} } \int_{0}^{\infty} \dd{k} {k}^2 \on{CT}(\vec{k})
\end{equation}

Several useful observations can be made on the choice of $\chi$.
\begin{itemize}
    \item By choosing $\chi = \frac{{k^*\pm}^2}{k^2}$, the Jacobian factor $k^2$ is canceled, leaving a simple $\frac{1}{k-k^*_{\pm}}$ integrand for the radial integral.
    \item If $\chi = \frac{{k^*_{\pm}}^2}{k^2} f(k)$ where $f$ is an even function, the real part of the integration vanishes, simplifying the computation.
\end{itemize}
Care must be taken when canceling the Jacobian factor this way. The parametrization of the subtracted integrand must match the parametrization of the counterterm, or other additional steps must be taken, to not introduce additional singularities into the integrand. In this thesis we will only consider the case where the counterterm is parametrized in the same way as the integrand.

For our purposes we choose a mask that is only nonzero in a finite region.
\begin{equation}
    \chi{\va{k}} = \begin{cases}
        \frac{{k^*_{\pm}}^2}{k^2} & k \in (\lambda_1, \lambda_2)\\
        0& \text{else}
    \end{cases}
\end{equation}
The radial integration is now straightforward.
\begin{equation}
   \int \dd{k} I_{\on{CT}}(k^*_{\pm}) = \int_{\lambda_1}^{\lambda_2} \dd{k} \frac{1}{k-k^*_{\pm}} = \ln(\frac{\lambda_2-k^*_{\pm}}{\lambda_1-k^*_{\pm}})
\end{equation}
In the special case, where $\lambda_1 = \Re k^*_{\pm} - \lambda$ and $\lambda_2 = \Re k_{\pm}^* + \lambda$, the real part vanishes due to symmetry around $\Re k^*_{\pm}$, leaving
\begin{equation}
    \int \dd{k} =  -i \qty(\pi + 2\arctan(\frac{\Im k^*_{\pm}}{\lambda}))
\end{equation}
This result is consistent with the result derived in~\cite{capatti_exposing_2023}.
Finally, including the angular integration and both $k^*_-$ and $k^*_+$, the full counterterm contribution of a single E-surface reads
\begin{equation}
    I_{\on{CT}} = \sum_{\pm} \frac{1}{2} \int_{S^2} \dd[2]{\hat{\vb{k}} } f(k^*) {k^*}^2 \ln(\frac{\lambda_2-k^*}{\lambda_1-k^*})
\end{equation}
The factor $\frac{1}{2}$ accounts for the fact that $\int_{S^2}$ integrates over the whole sphere thus integrating over two opposing hemispheres.
\subsection{Threshold subtraction in the actual integrand}
In my examples we will only ever choose parametrizations centered around the origin $k^0 = \va{0}$. Without loss of generality we can reorder $\vb{q}_i$ such that $\vb{q}_1 < \vb{q}_2 < \vb{q}_3$. Thus the existence condition derived in Section~\ref{sec:e_surface_exist} is fulfilled for the following E-surfaces:
\begin{itemize}
    \item $\ELIPT{1}{2}$
    \item $\ELIPT{2}{3}$
    \item $\ELIPT{1}{3}$
\end{itemize}
Each of these requires it's own counterterm.
As an example, consider $\ELIPT{1}{2}$.
First we need to collect all of the terms in the integrand containing the relevant E-surface $\ELIPT{1}{2}$. 
This factor corresponds exactly to the function $f$ introduced in Section~\ref{sec:thresh}.
\begin{equation}
    \ELIPT{1}{2} \qty{1}{{(4\pi)}^3}\frac{1}{E_1E_2E_3} \qty(\frac{1}{\ELIPT{1}{3}}+\frac{1}{\ELIPT{2}{3}}) = \ELIPT{3}{2} \, f(\va{k})
\end{equation}
The counterterm removing the pole at either $k^*_+$ or $k^*_-$ is therefore given by
\begin{equation}
    \on{CT}_{12}(k^*_\pm) = \chi(k-k^*) \, f(\va{k}) \,\frac{1}{\eta'_{12}} \,  \frac{1}{k-k^*_\pm}
\end{equation}
where $\eta_{12}'$ is the radial derivative of $\ELIPT{1}{2}$ evaluated at the pole.
The same procedure applies to $\ELIPT{2}{3}$ and $\ELIPT{1}{3}$
The integral is then finally given by
\begin{equation}
    I = \int \dd[3]{k} \qty(\mathcal{I}_{CFF} - \sum_{\pm} \sum_{(i,j) \in \{(1,2),(2,3),(1,3)\}} \on{CT}_{ij}(k^*_\pm)) + \sum_{\pm} \sum_{(i,j) \in \{(1,2),(2,3),(1,3)\}} I_{\on{CT}_{ij}}(k^*_\pm)
\end{equation}

\section{Numerical Integration}
The numerical integration is performed with the Python API of the \texttt{symbolica} library~\cite{symbolica}.
\texttt{Symbolica} supports compilation of symbolic expressions into optimized assembly code for fast evaluation. It also provides an efficient implementation of the \texttt{VEGAS} Monte Carlo integration algorithm, which is well suited for this problem.
\subsection{Parametrization}
\texttt{Symbolica} provides efficient sampling in $N$-dimensional hypercubes.
There are many reasonable choices to map from the unit hypercube to hemispherical coordinates. I choose the following mapping from unit hypercubes to the relevant parametrizations with their respective Jacobians $J$.

\subsubsection{parametrization of the unit hemisphere}
\begin{equation}
    \vb{\hat{k}}(u, v) = 
    \begin{pmatrix}
        \sin(\phi) \cos(2 \pi w)\\
        \sin(\phi) \sin(2 \pi w)\\
        \cos(\phi)
    \end{pmatrix}
\end{equation}
with
\begin{equation}
    \cos(\phi) = 1 - 2 v\qquad
    \sin(\phi) = \sqrt{1-\cos^2(\phi)}\qquad
    J_{\vb{\hat{k}}} = 2 \pi
\end{equation}

\subsubsection{Spherical Parametrization}
\begin{equation}
    \va{k}(u, v, w) = r \vb{\hat{k}}(u, v)
\end{equation}

\begin{equation}
    \tilde{u} = 2\,u-1 \qquad
    r = \tilde{u} - \frac{1}{\tilde{u}} \qquad
    J = \frac{r^2 \, J_{\vb{\hat{k}}}}{{(1-\tilde{u})}^2}
\end{equation}

\section{Results}

Plots for threshold subtraction.

Plots of numerical stability with relative error (dimensionless)

\section{Discussion}


\printbibliography
\appendix
\section{Rewriting the LTD expression}\label{app:cff}
We now want to rewrite this expression into a form that is better suited for
numerical integration. For this we will make heavy use of the following
identities.
\begin{equation}
    \frac{1}{xy} = \frac{1}{x-y}\qty(\frac{1}{y}-\frac{1}{x})
\end{equation}
The following relations also hold for the $\eta$ coefficients.
\begin{align}
    \HYPER{i}{j}                           & = -\HYPER{j}{i}               \label{eq:hyper_antisym} \\
    \ELIPT{i}{j}-\HYPER{i}{j}              & = 2E_j           \label{eq:elipt_min_hyper}            \\
    \ELIPT{i}{k}-\ELIPT{j}{k}              & = \HYPER{i}{j}   \label{eq:elipt_min_elipt}            \\
    \HYPER{i}{j}+\HYPER{j}{k}+\HYPER{k}{i} & = 0 \label{eq:hyper_cycle}
\end{align}
To start we can notice the following useful identity
\begin{align}
    \frac{1}{\ELIPT{i}{j}\HYPER{i}{j}} & = \frac{1}{2E_j}\qty(\frac{1}{\HYPER{i}{j}} - \frac{1}{\ELIPT{i}{j}}) \label{eq:rewrite_1}
\end{align}
Applying Eq.~\eqref{eq:rewrite_1} to the integrand we get
%\begin{align}
%I(q,p,m_\psi) =& \int \frac{\dd[3]{k}}{{(2\pi)}^3} \frac{1}{(2E_1)(2E_2)(2E_3)} \\
%&  \sum_{i=1}^{3} \prod_{i\neq j}\left(\frac{1}{\HYPER{i}{j}}-\frac{i}{\ELIPT{i}{j}}\right)
%\end{align}
\newcommand{\HMFRAC}[2]{\left(\frac{1}{\HYPER{#1}{#2}}-\frac{1}{\ELIPT{#1}{#2}}\right)}
\begin{align*}
    I(q,p,m_\psi) = & \int \frac{\dd[3]{k}}{{(2\pi)}^3} \frac{1}{(2E_1)(2E_2)(2E_3)} \Bigg[ \\
                    & \phantom{{}+{}}
    \HMFRAC{1}{2}\HMFRAC{1}{3}                                                              \\
                    & +\HMFRAC{2}{1}\HMFRAC{2}{3}                                           \\
                    & +\HMFRAC{3}{1}\HMFRAC{3}{2}
    \Bigg]
\end{align*}
This can be expanded
\begin{align*}
    I(q,p,m_\psi)
     & = \int \frac{\dd[3]{k}}{{(4\pi)}^3} \frac{1}{E_1 E_2 E_3} \Bigg[ \\
     & \frac{1}{\HYPER{1}{2}\HYPER{1}{3}}
        - \frac{1}{\HYPER{1}{2}\ELIPT{1}{3}}
        - \frac{1}{\ELIPT{1}{2}\HYPER{1}{3}}
    + \frac{1}{\ELIPT{1}{2}\ELIPT{1}{3}}                                \\
     & + \frac{1}{\HYPER{2}{1}\HYPER{2}{3}}
        - \frac{1}{\HYPER{2}{1}\ELIPT{2}{3}}
        - \frac{1}{\ELIPT{2}{1}\HYPER{2}{3}}
    + \frac{1}{\ELIPT{2}{1}\ELIPT{2}{3}}                                \\
     & + \frac{1}{\HYPER{3}{1}\HYPER{3}{2}}
        - \frac{1}{\HYPER{3}{1}\ELIPT{3}{2}}
        - \frac{1}{\ELIPT{3}{1}\HYPER{3}{2}}
        + \frac{1}{\ELIPT{3}{1}\ELIPT{3}{2}}
        \Bigg]
\end{align*}
and then rearranged
\begin{align*}
    I(q,p,m_\psi)
     & =  \int \frac{\dd[3]{k}}{{(4\pi)}^3} \frac{1}{E_1E_2E_3} \Bigg[           \\
     & \phantom{{}+{}}
    \frac{1}{\ELIPT{1}{2}\ELIPT{1}{3}}
    + \frac{1}{\ELIPT{2}{1}\ELIPT{2}{3}}
    + \frac{1}{\ELIPT{3}{1}\ELIPT{3}{2}}                                         \\
     &
    -\frac{1}{\HYPER{1}{2}}\qty(\frac{1}{\ELIPT{1}{3}} - \frac{1}{\ELIPT{2}{3}})
    -\frac{1}{\HYPER{3}{1}}\qty(\frac{1}{\ELIPT{3}{2}} - \frac{1}{\ELIPT{1}{2}})
    -\frac{1}{\HYPER{2}{3}}\qty(\frac{1}{\ELIPT{2}{1}} - \frac{1}{\ELIPT{3}{1}}) \\
     &
    + \frac{1}{\HYPER{1}{2}\HYPER{1}{3}}
    + \frac{1}{\HYPER{2}{1}\HYPER{2}{3}}
    + \frac{1}{\HYPER{3}{1}\HYPER{3}{2}}
    \Bigg]
\end{align*}
%We will now show that the purely hyperbolic terms vanish, for this we need
%\begin{align}
%    \frac{1}{\HYPER{i}{j}\HYPER{i}{k}} &= \frac{1}{\HYPER{j}{k}}\qty(\frac{1}{\HYPER{i}{j}} - \frac{1}{\HYPER{i}{k}})\\
%    &= \frac{\HYPER{i}{k}-\HYPER{i}{j}}{\HYPER{i}{j}\HYPER{i}{j}\HYPER{i}{k}} \label{eq:hyper_comm_denom}
%\end{align}
%If we apply Eq.~\eqref{eq:hyper_comm_denom} to each term we get
%\begin{align}
%    \frac{-\HYPER{1}{2}-\HYPER{1}{3}-\HYPER{2}{1}-\HYPER{2}{3}-\HYPER{3}{1}-\HYPER{3}{2}}{\HYPER{1}{2}\HYPER{2}{3}\HYPER{3}{1}}
%    &=
%    \frac{\HYPER{1}{2}-\HYPER{1}{2}+\HYPER{3}{1}-\HYPER{1}{3}+\HYPER{2}{3}-\HYPER{3}{2}}{\HYPER{1}{2}\HYPER{2}{3}\HYPER{3}{1}} = 0
%\end{align}
We will now show that the purely hyperbolic terms vanish, for this we can
simply put the on a common denominator and apply Eq.~\eqref{eq:hyper_cycle}.
\begin{equation}
    \frac{\HYPER{1}{2}+\HYPER{2}{3}+\HYPER{3}{1}}{\HYPER{1}{3}\HYPER{2}{3}\HYPER{3}{1}} = 0
\end{equation}
We now only need to take care of the mixed terms. For this we can use the identity
\begin{align}
    -\frac{1}{\HYPER{i}{j}} \qty(\frac{1}{\ELIPT{i}{k}}-\frac{1}{\ELIPT{j}{k}})
     & = \frac{1}{\HYPER{i}{j}} \qty(\frac{\ELIPT{i}{k}-\ELIPT{j}{k}}{\ELIPT{i}{k}\ELIPT{j}{k}}) \\
     & = \frac{1}{\ELIPT{i}{k}\ELIPT{j}{k}} \label{eq:rewrite_2}
\end{align}
Here we used Eq.~\eqref{eq:elipt_min_elipt} in the last step.
Applying Eq.~\eqref{eq:rewrite_2} to each mixed term we get our final result.
\begin{align*}
    I(q,p,m_\psi)
     & =  \int \frac{\dd[3]{k}}{{(4\pi)}^3} \frac{1}{E_1E_2E_3} \Bigg[ \\
     & \frac{1}{\ELIPT{2}{1}\ELIPT{3}{1}}
        + \frac{1}{\ELIPT{1}{2}\ELIPT{1}{3}}
        + \frac{1}{\ELIPT{1}{2}\ELIPT{3}{2}}
        + \frac{1}{\ELIPT{2}{1}\ELIPT{2}{3}}
        + \frac{1}{\ELIPT{1}{3}\ELIPT{2}{3}}
        + \frac{1}{\ELIPT{3}{1}\ELIPT{3}{2}}
        \Bigg]
\end{align*}

\section{OneLOopBridge}\label{app:oneloopbridge}

\textit{OneLOopBridge} provides a lightweight Rust and Python interface to the \textit{OneLOop} Fortran library, providing access to scalar one-loop integrals with up to four external legs. The package was developed to offer a safe, idiomatic, and reproducible way to evaluate loop integrals.

The underlying \textit{OneLOop} library implements analytic expressions for
dimensionally regularized scalar one-loop integrals, returning their Laurent
expansion in the dimensional regulator $\varepsilon = (4-d)/2$.
\textit{OneLOopBridge} exposes these routines by providing save Rust bindings
and a Python interface. In addition to the computation of integrals, and
interface is provided to access the most important internal settings of
\textit{OneLOop}, namely:
\begin{itemize}
    \item the renormalization scale,
    \item the on-shell threshold,
    \item and the internal logging level.
\end{itemize}

All integral routines return a \texttt{ResultOLO} object containing the
coefficients of the Laurent expansion,
\begin{equation}
    I = \varepsilon^{-2} I_{-2} + \varepsilon^{-1} I_{-1} + I_{0} + \mathcal{O}(\varepsilon)
\end{equation}
where finite integrals automatically yield vanishing divergent coefficients. The coefficients are accessible individually. A global conversion factor \texttt{TO\_FEYNMAN} is provided to translate results to the conventional Feynman integral normalization.

\subsection{Rust Interface}

The Rust API mirrors the structure of the underlying integrals and emphasizes
explicitness. Each function corresponds to a specific configuration and takes
the relevant kinematic invariants and squared masses as arguments. Complex
masses are supported via \texttt{num\_complex::Complex64}.

Typical usage involves computing Lorentz invariants externally and passing them
to the wrapper functions. The returned \texttt{ResultOLO} struct provides
accessor methods for the Laurent coefficients.

The package can simply be installed using \texttt{cargo}

\begin{minted}{bash}
cargo add --git https://github.com/SecretGmG/OneLOopBridge
\end{minted}

\subsection{Python Interface}
Python bindings are generated using \texttt{maturin} and \texttt{PyO3},
exposing the same functionality as the rust interface. The package includes
python stub files, allowing the users linting and type checking to integrate
the library.

To install and compile the python bindings it is necessary to download the
repository and compile it manually. After that maturin can be run to install the bindings
into an active python environment. This can be done by running the following
inside an active python environment. Note that this will change the current
working directory.
\begin{minted}{bash}
git clone https://github.com/SecretGmG/OneLOopBridge.git
cd OneLOopBridge && make -f Makefile develop
\end{minted}



\end{document}