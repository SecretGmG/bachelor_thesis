The goal is to prepare the scalar triangle integral $I$ for numerical integration by integrating over the $k^0$ component analytically.
\begin{equation}
    I = \int \frac{\dd[4]{k}}{{(2\pi)}^4} \frac{i}{D_1D_2D_3} \label{eq:integral}
\end{equation}
with
\begin{equation}
    D_i = {(k-q_i)}^2-m^2+i\varepsilon \label{eq:denom}
\end{equation}.
Note that for the external momenta $p_1$ and $p_2$ we have
\begin{align}
    q_1 = l\\
    q_2 = l - p_1\\
    q_3 = l + p_2
\end{align}
where $l$ can be chosen freely.
\subsection{Loop Tree Duality}
We will perform the $k^0$ integration with the cauchy theorem, and want to close the contour such that only the poles coming from the principle square root are included. If we close the contour with a semi-circle, it is easy to see that it's contribution vanishes as $r \to \infty$
\begin{equation}
    \int_{0}^{-\pi} \dd{\phi} \frac{ir}{D_1D_2D_3} \sim \int_{0}^{-\pi} \dd{\phi} \order{r^{-5}}
\end{equation}

The poles of the integrand from Eq.~\eqref{eq:integral} in $k^0$ are found to be
\begin{equation}
    k^0 = q_i^0 \pm \qty(\sqrt{{\qty(\va{k}-\va{q}_i)}^2+m^2}-i\varepsilon)
\end{equation}
Let's call the value of the principal square root $E_i$.
\begin{equation}
    E_i = \sqrt{{\qty(\va{k}-\va{q}_i)}^2 + m^2}
\end{equation}
Notice that the principle square root preserves the sign of the imaginary part of the argument. Thus we need to close the contour below if $\Im{m^2}\leq0$ and above otherwise. We can now apply cauchy's theorem. From now on we will assume that $\Im{m^2}\leq$ to keep the notation simple, the other case can be obtained by multiplying with $-1$.
\begin{equation}
    I = \int \frac{\dd[4]{k}}{{(2\pi)}^4} 2\pi i \sum_{i= 1}^{3} \On{Res}_{k^0 = q^0_i + E_i}\qty[\frac{i}{D_1D_2D_3}] \label{eq:cauchy}
\end{equation}
Let's compute the residue.
\begin{equation}
    \On{Res}_{k^0 = q^0_i + E_i}\qty[\frac{i}{D_1D_2D_3}] = \eval{\frac{i}{D_i'}\prod_{\neq j} \frac{1}{D_j}}_{k^0=E_i}= \frac{i}{2E_i} \prod_{i\neq j} \eval{\frac{1}{D_j}}_{k^0=E_i} \label{eq:residue}
\end{equation}
We can also introduce the following simplification.

\newcommand{\ELIPT}[2]{\eta_{#1#2}^{++}}
\newcommand{\HYPER}[2]{\eta_{#1#2}^{+-}}
\begin{align}
    \eval{D_j}_{k^0=E_i} &= {\qty(q^0_i+E_i-q^0_j)}^2 - \overbrace{{\qty(\va{k}-\va{q}_j)}^2+m^2}^{E_j^2}\\
    &= E_i^2+{q_i^0}^2+{q_j^0}^2-2q_i^0q_j^0+2q_i^0E_i - 2q_j^0E_i-E_j^2\\
    &= \underbrace{\left(E_i+E_j+q_i^0-q_j^0\right)}_{\ELIPT{i}{j}}\underbrace{\left(E_i-E_j+q_i^0-q_j^0\right)}_{\HYPER{i}{j}} \label{eq:surfaces}
\end{align}
inserting the results from Eq.~\eqref{eq:surfaces} and Eq.~\eqref{eq:residue} into Eq.~\eqref{eq:cauchy} we get the expression.
\newcommand{\BLTD}[3]{2E_{#1}\ELIPT{#1}{#2}\HYPER{#1}{#2}\ELIPT{#1}{#3}\HYPER{#1}{#3}}
\begin{equation}
    I =  \int \frac{\dd[3]{k}}{{(2\pi)}^3}\qty(\frac{1}{\BLTD{1}{2}{3}}+\frac{1}{\BLTD{2}{1}{3}}+\frac{1}{\BLTD{3}{1}{2}})
\end{equation}
\subsection{Rewriting the LTD expression}
We now want to rewrite this expression into a form that is better suited for numerical integration.
For this we will make heavy use of the following identities.
\begin{equation}
    \frac{1}{xy} = \frac{1}{x-y}\qty(\frac{1}{y}-\frac{1}{x})
\end{equation}
The following relations also hold for the $\eta$ coefficients.
\begin{align}
    \HYPER{i}{j} &= -\HYPER{j}{i}               \label{eq:hyper_antisym}\\
    \ELIPT{i}{j}-\HYPER{i}{j} &= 2E_j           \label{eq:elipt_min_hyper}\\
    \ELIPT{i}{k}-\ELIPT{j}{k} &= \HYPER{i}{j}   \label{eq:elipt_min_elipt}\\
    \HYPER{i}{j}+\HYPER{j}{k}+\HYPER{k}{i} &= 0 \label{eq:hyper_cycle}
\end{align}
To start we can notice the following useful identity
\begin{align}
    \frac{1}{\ELIPT{i}{j}\HYPER{i}{j}} &= \frac{1}{2E_j}\qty(\frac{1}{\HYPER{i}{j}} - \frac{1}{\ELIPT{i}{j}}) \label{eq:rewrite_1}
\end{align}
Applying Eq.~\eqref{eq:rewrite_1} to the integrand we get
%\begin{align}
%I(q,p,m_\psi) =& \int \frac{\dd[3]{k}}{{(2\pi)}^3} \frac{1}{(2E_1)(2E_2)(2E_3)} \\
%&  \sum_{i=1}^{3} \prod_{i\neq j}\left(\frac{1}{\HYPER{i}{j}}-\frac{i}{\ELIPT{i}{j}}\right)
%\end{align}
\newcommand{\HMFRAC}[2]{\left(\frac{1}{\HYPER{#1}{#2}}-\frac{1}{\ELIPT{#1}{#2}}\right)}
\begin{align*}
I(q,p,m_\psi) =&  \int \frac{\dd[3]{k}}{{(2\pi)}^3} \frac{1}{(2E_1)(2E_2)(2E_3)} \Bigg[\\
      &  \phantom{{}+{}}
          \left(\HMFRAC{1}{2}\right)\left(\HMFRAC{1}{3}\right)\\
      &  +\left(\HMFRAC{2}{1}\right)\left(\HMFRAC{2}{3}\right)\\
      &  +\left(\HMFRAC{3}{1}\right)\left(\HMFRAC{3}{2}\right)
      \Bigg]
\end{align*}
This can be expanded and rearranged to
%\begin{align*}
%I(q,p,m_\psi)
%&= \int \int \frac{\dd[3]{k}}{{(2\pi)}^3} \frac{1}{(2E_1)(2E_2)(2E_3)} \Bigg[\\
%&  \frac{1}{\HYPER{1}{2}\HYPER{1}{3}}
% - \frac{1}{\HYPER{1}{2}\ELIPT{1}{3}}
% - \frac{1}{\ELIPT{1}{2}\HYPER{1}{3}}
% + \frac{1}{\ELIPT{1}{2}\ELIPT{1}{3}}\\
%&+ \frac{1}{\HYPER{2}{1}\HYPER{2}{3}}
% - \frac{1}{\HYPER{2}{1}\ELIPT{2}{3}}
% - \frac{1}{\ELIPT{2}{1}\HYPER{2}{3}}
% + \frac{1}{\ELIPT{2}{1}\ELIPT{2}{3}}\\
%&+ \frac{1}{\HYPER{3}{1}\HYPER{3}{2}}
% - \frac{1}{\HYPER{3}{1}\ELIPT{3}{2}}
% - \frac{1}{\ELIPT{3}{1}\HYPER{3}{2}}
% + \frac{1}{\ELIPT{3}{1}\ELIPT{3}{2}}
%\Bigg]
%\end{align*}
%This can be rearranged to
\begin{align*}
I(q,p,m_\psi)
&=  \int \frac{\dd[3]{k}}{{(2\pi)}^3} \frac{1}{(2E_1)(2E_2)(2E_3)} \Bigg[\\
&\phantom{{}+{}}
  \frac{1}{\ELIPT{1}{2}\ELIPT{1}{3}}
+ \frac{1}{\ELIPT{2}{1}\ELIPT{2}{3}}
+ \frac{1}{\ELIPT{3}{1}\ELIPT{3}{2}}\\
&
-\frac{1}{\HYPER{1}{2}}\qty(\frac{1}{\ELIPT{1}{3}} - \frac{1}{\ELIPT{2}{3}})
-\frac{1}{\HYPER{3}{1}}\qty(\frac{1}{\ELIPT{3}{2}} - \frac{1}{\ELIPT{1}{2}})
-\frac{1}{\HYPER{2}{3}}\qty(\frac{1}{\ELIPT{2}{1}} - \frac{1}{\ELIPT{3}{1}})\\
&
+ \frac{1}{\HYPER{1}{2}\HYPER{1}{3}}
+ \frac{1}{\HYPER{2}{1}\HYPER{2}{3}}
+ \frac{1}{\HYPER{3}{1}\HYPER{3}{2}}
\Bigg]
\end{align*}
%We will now show that the purely hyperbolic terms vanish, for this we need
%\begin{align}
%    \frac{1}{\HYPER{i}{j}\HYPER{i}{k}} &= \frac{1}{\HYPER{j}{k}}\qty(\frac{1}{\HYPER{i}{j}} - \frac{1}{\HYPER{i}{k}})\\
%    &= \frac{\HYPER{i}{k}-\HYPER{i}{j}}{\HYPER{i}{j}\HYPER{i}{j}\HYPER{i}{k}} \label{eq:hyper_comm_denom}
%\end{align}
%If we apply Eq.~\eqref{eq:hyper_comm_denom} to each term we get
%\begin{align}
%    \frac{-\HYPER{1}{2}-\HYPER{1}{3}-\HYPER{2}{1}-\HYPER{2}{3}-\HYPER{3}{1}-\HYPER{3}{2}}{\HYPER{1}{2}\HYPER{2}{3}\HYPER{3}{1}}
%    &=
%    \frac{\HYPER{1}{2}-\HYPER{1}{2}+\HYPER{3}{1}-\HYPER{1}{3}+\HYPER{2}{3}-\HYPER{3}{2}}{\HYPER{1}{2}\HYPER{2}{3}\HYPER{3}{1}} = 0
%\end{align}
We will now show that the purely hyperbolic terms vanish,
for this we can simply put the on a common denominator and apply Eq.~\eqref{eq:hyper_cycle}.
\begin{equation}
    \frac{\HYPER{1}{2}+\HYPER{2}{3}+\HYPER{3}{1}}{\HYPER{1}{3}\HYPER{2}{3}\HYPER{3}{1}} = 0
\end{equation}
We now only need to take care of the mixed terms. For this we can use the identity
\begin{align}
    -\frac{1}{\HYPER{i}{j}} \qty(\frac{1}{\ELIPT{i}{k}}-\frac{1}{\ELIPT{j}{k}})
&= \frac{1}{\HYPER{i}{j}} \qty(\frac{\ELIPT{i}{k}-\ELIPT{j}{k}}{\ELIPT{i}{k}\ELIPT{j}{k}})\\
&= \frac{1}{\ELIPT{i}{k}\ELIPT{j}{k}} \label{eq:rewrite_2}
\end{align}
Here we used Eq.~\eqref{eq:elipt_min_elipt} in the last step.
Applying Eq.~\eqref{eq:rewrite_2} to each mixed term we get our final result.
\begin{align*}
I(q,p,m_\psi)
&=  \int \frac{\dd[3]{k}}{{(2\pi)}^3} \frac{1}{(2E_1)(2E_2)(2E_3)} \Bigg[\\
  &\frac{1}{\ELIPT{2}{1}\ELIPT{3}{1}}
+ \frac{1}{\ELIPT{1}{2}\ELIPT{1}{3}}
+ \frac{1}{\ELIPT{1}{2}\ELIPT{3}{2}}
+ \frac{1}{\ELIPT{2}{1}\ELIPT{2}{3}}
+ \frac{1}{\ELIPT{1}{3}\ELIPT{2}{3}}
+ \frac{1}{\ELIPT{3}{1}\ELIPT{3}{2}}
\Bigg]
\end{align*}