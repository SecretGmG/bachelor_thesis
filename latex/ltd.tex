\section{ltd}

The goal is to prepare the scalar triangle integral $I$ for numerical integration by integrating over the $k^0$ component analytically.
\begin{equation}
    I = \int \frac{\dd[4]{k}}{{(2\pi)}^4} \frac{i}{D_1D_2D_3} \label{eq:integral}
\end{equation}
with
\begin{equation}
    D_i = {(k-q_i)}^2-m^2+i\varepsilon \label{eq:denom}
\end{equation}.
Note that for the external momenta $p_1$ and $p_2$ we have
\begin{align}
    q_1 = l\\
    q_2 = l - p_1\\
    q_3 = l + p_2
\end{align}
where $l$ can be chosen freely.
\subsection{Loop Tree Duality}
We will perform the $k^0$ integration with the cauchy theorem, and want to close the contour such that only the poles coming from the principle square root are included. If we close the contour with a semi-circle, it is easy to see that it's contribution vanishes as $r \to \infty$
\begin{equation}
    \int_{0}^{-\pi} \dd{\phi} \frac{ir}{D_1D_2D_3} \sim \int_{0}^{-\pi} \dd{\phi} \order{r^{-5}}
\end{equation}

The poles of the integrand from Eq.~\eqref{eq:integral} in $k^0$ are found to be
\begin{equation}
    k^0 = q_i^0 \pm \qty(\sqrt{{\qty(\va{k}-\va{q}_i)}^2+m^2}-i\varepsilon)
\end{equation}
Let's call the value of the principal square root $E_i$.
\begin{equation}
    E_i = \sqrt{{\qty(\va{k}-\va{q}_i)}^2 + m^2}
\end{equation}
Notice that the principle square root preserves the sign of the imaginary part of the argument. Thus we need to close the contour below if $\Im{m^2}\leq0$ and above otherwise. We can now apply cauchy's theorem. From now on we will assume that $\Im{m^2}\leq$ to keep the notation simple, the other case can be obtained by multiplying with $-1$.
\begin{equation}
    I = \int \frac{\dd[4]{k}}{{(2\pi)}^4} 2\pi i \sum_{i= 1}^{3} \On{Res}_{k^0 = q^0_i + E_i}\qty[\frac{i}{D_1D_2D_3}] \label{eq:cauchy}
\end{equation}
Let's compute the residue.
\begin{equation}
    \On{Res}_{k^0 = q^0_i + E_i}\qty[\frac{i}{D_1D_2D_3}] = \eval{\frac{i}{D_i'}\prod_{\neq j} \frac{1}{D_j}}_{k^0=E_i}= \frac{i}{2E_i} \prod_{i\neq j} \eval{\frac{1}{D_j}}_{k^0=E_i} \label{eq:residue}
\end{equation}
We can also introduce the following simplification.

\begin{align}
    \eval{D_j}_{k^0=E_i} &= {\qty(q^0_i+E_i-q^0_j)}^2 - \overbrace{{\qty(\va{k}-\va{q}_j)}^2+m^2}^{E_j^2}\\
    &= E_i^2+{q_i^0}^2+{q_j^0}^2-2q_i^0q_j^0+2q_i^0E_i - 2q_j^0E_i-E_j^2\\
    &= \underbrace{\left(E_i+E_j+q_i^0-q_j^0\right)}_{\ELIPT{i}{j}}\underbrace{\left(E_i-E_j+q_i^0-q_j^0\right)}_{\HYPER{i}{j}} \label{eq:surfaces}
\end{align}
inserting the results from Eq.~\eqref{eq:surfaces} and Eq.~\eqref{eq:residue} into Eq.~\eqref{eq:cauchy} we get the expression.
\begin{equation}
    I =  \int \frac{\dd[3]{k}}{{(2\pi)}^3}\qty(\frac{1}{\BLTD{1}{2}{3}}+\frac{1}{\BLTD{2}{1}{3}}+\frac{1}{\BLTD{3}{1}{2}})
\end{equation}