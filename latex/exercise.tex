\section{Theoretical Exercise}

A simplified version of the integral is:
\begin{align}
I(q,p,m_\psi)&=\int\frac{d^4k}{(2\pi)^4}
\frac{i}{\left(k^2-m_\psi^2+i\epsilon\right)\left((k-q)^2-m_\psi^2+i\epsilon\right)\left((k+p)^2-m_\psi^2+i\epsilon\right)}\\
&=\int\frac{d^4k}{(2\pi)^4}
i\prod_{i=1}^{3}\frac{1}{\left((k-q_i)^2-m_\psi^2+i\epsilon\right)}
\end{align}
with $q_i = (0,-q,p)_i$.

The pole conditions for each term in the product becomes
\begin{equation}
(k+q_i)^2 = m_\psi^2-i\epsilon \quad\implies\quad k^0+q_i^0 = \pm\sqrt{(\vec{k}+\vec{q}_i)^2+m_\psi^2} = \pm E_i
\end{equation}
Choosing to close the contour below we can then apply the residue theorem.
The integration over the semicircle vanishes in the limit $r\to\infty$ because the integrand is $\propto r^{-6}$.
\begin{equation}
    \int_{\text{semi circle}} dk^0 r^{-6} \propto \int_{\pi} d\theta r^{-5} \propto r^{-5} \overset{r\to\infty}{\longrightarrow} 0
\end{equation}
The residues can be computed with l'Hospitals rule
\begin{align}
\Op{Res}_{E_i=k^0+q_i^0}
    &= \left. \frac{1}{2 E_i} \prod_{j\neq i} \frac{1}{ \left((k1q_i)^2-m_\psi^2+i\epsilon\right)} \right|_{k^0+q_i^0 = +E_1}\\
    &= \int_{k^0} d k \, \theta(k^0) \, \delta\left(k^2 - m_\psi^2\right) \prod_{j\neq i}\frac{1}{\left((k+q_j)^2-m_\psi^2+i\epsilon\right)}
\end{align}
Notice that the application of the delta exactly reproduces the derivatives, and the heaviside function ensures that the correct pole is selected.
If we now define the shorthand notation:
\begin{equation}
    \delta^{+}\left((k+q_i)^2 - m_\psi^2 \right) = \theta(k^01q_i^0) \, \delta\left((k+q_i)^2 - m_\psi^2\right) E_i
\end{equation}
We can now apply the residue theorem, notice that we pick up an extra $-$ sign due to the clockwise direction of the semicircle
\begin{align*}
   I(q,p,m_\psi) &= \int\frac{d^3k}{(2\pi)^4} \int dk^0 i \prod_{i=1}^{3}\frac{1}{\left((k-q_i)^2-m_\psi^2+i\epsilon\right)}\\
    &= \int\frac{d^3k}{(2\pi)^4} (2\pi i) \, i \left( - \sum_i \Op{Res}_{E_i=k^0+q_i^0} \right)\\
    &= \int\frac{d^4k}{(2\pi)^3}
\frac{\delta^{+}\left( k^2 - m_\psi^2 \right)+\delta^{+}\left( (k+q)^2 - m_\psi^2 \right)+\delta^{+}\left( (k+p)^2 - m_\psi^2 \right)}{\left(k^2-m_\psi^2+i\epsilon\right)\left((k+q)^2-m_\psi^2+i\epsilon\right)\left((k+p)^2-m_\psi^2+i\epsilon\right)}
\end{align*}
The execution of the $k^0$ integral is now a matter of inserting the correct values, we can introduce the notation $\bar{\eta}^{\pm_1\pm_2}_{i,j} =\pm_1E_i \pm_2E_j$ to keep the result shorter. This is procedure is especially easy (but tedious) when using the intermediary results from the previous step.
\begin{align*}
    I(q,p,m_\psi)
    &= \int\frac{d^3k}{(2\pi)^3}
    \Bigg[\\
    &\quad \phantom{+} \frac{1}{2E_1}
    \frac{1}{(\bar{\eta}^{++}_{12} - q^0)(\bar{\eta}^{+-}_{12} - q^0)}
    \frac{1}{(\bar{\eta}^{++}_{13} + p^0)(\bar{\eta}^{+-}_{13} + p^0)} \\
    &\quad +
    \frac{1}{(\bar{\eta}^{++}_{21} + q^0)(\bar{\eta}^{+-}_{21} + q^0)}
    \frac{1}{2E_2}
    \frac{1}{(\bar{\eta}^{++}_{23} + p^0 + q^0)(\bar{\eta}^{+-}_{23} + p^0 + q^0)} \\
    &\quad +
    \frac{1}{(\bar{\eta}^{++}_{31} - p^0)(\bar{\eta}^{+-}_{31} - p^0)}
    \frac{1}{(\bar{\eta}^{++}_{32} - p^0 - q^0)(\bar{\eta}^{+-}_{32} - p^0 - q^0)}
    \frac{1}{2E_3}
    \Bigg].
\end{align*}
We can also absorb the Energy shifts into the $\eta$ coefficients by introducing the notation $\eta^{\pm_1\pm_2}_{i,j} =\pm_1E_i \pm_2E_j \mp_1 (q^0_i-q^0_j)$ and $q_i = (0,-q, p)$.
\begin{equation}
    I(q,p,m_\psi)
    = \int\frac{d^3k}{(2\pi)^3}
    \Bigg[
    \frac{1}{2E_1}
    \frac{1}{\eta^{++}_{12}\,\eta^{+-}_{12}}
    \frac{1}{\eta^{++}_{13}\,\eta^{+-}_{13}}
     +
    \frac{1}{2E_2}
    \frac{1}{\eta^{++}_{21}\,\eta^{+-}_{21}}
    \frac{1}{\eta^{++}_{23}\,\eta^{+-}_{23}}
     +
    \frac{1}{2E_3}
    \frac{1}{\eta^{++}_{31}\,\eta^{+-}_{31}}
    \frac{1}{\eta^{++}_{32}\,\eta^{+-}_{32}}
    \Bigg]. \label{eq:ltd}
\end{equation}
We will now take a closer look at the singularities of this integral. We have a singularity exactly when one of the $\eta$ coefficients is $0$. It makes sense to split into 2 cases:
\begin{align}
    \eta^{++}_{ij} = E_i + E_j - (q^0_i - q^0_j)\\
    \eta^{+-}_{ij} = E_i - E_j + (q^0_i - q^0_j)\\
\end{align}
You can find an example here \url{https://www.desmos.com/3d/7gzgrreciz}
First try finding an existence condition that any zero exists for $\eta^{++}$
\begin{align}
    0 &= E_i + E_j - (q^0_i - q^0_j)\\
    &= \sqrt{(\vec{k}+\vec{q}_i)^2+m_\psi^2} + \sqrt{(\vec{k}+\vec{q}_j)^2+m_\psi^2} - (q^0_i - q^0_j)\\
\end{align}
We can now simplify this expression by introducing $\vec{l} = \vec{k}+\vec{q}_i$
\begin{align}
    &= \sqrt{\vec{l}^2+m_\psi^2} + \sqrt{(\vec{l}+\vec{q}_j-\vec{q}_i)^2+m_\psi^2} - (q^0_i - q^0_j)
\end{align}
For sufficiently large values of $\vec{l}$ this is always positive. It is also easy to show, e.g. by taking the derivative, that the minimum of this expression is at $\vec{l} = \frac{1}{2} (\vec{q}_j-\vec{q}_i)$. Since the equation is continuous there exists a zero iff the minimum is $\leq 0$.
\begin{align}
    \sqrt{(\frac{1}{2} (\vec{q}_j-\vec{q}_i))^2+m_\psi^2} + \sqrt{(\frac{1}{2}(\vec{q}_j-\vec{q}_i))^2+m_\psi^2} - (q^0_i - q^0_j) &\leq 0\\
    \implies\quad (\vec{q}_j-\vec{q}_i)^2 + 4 m_\psi^2 &\leq (q^0_i - q^0_j)^2\\
    \implies\quad (q^0_i - q^0_j)^2 - (\vec{q}_j-\vec{q}_i)^2 -  &\geq 4 m_\psi^2\\
    m_S \geq 2 m_\psi
\end{align}
Luckily the remaining $\eta^{+-}$ singularities all cancel pairwise, we will show this by repeatedly applying the partial fractioning identity to \eqref{eq:ltd}
\begin{equation}
    \frac{1}{xy} = \frac{1}{x-y}\left(\frac{1}{y}-\frac{1}{x}\right)
\end{equation}
First notice however, that
\begin{align}
    \eta^{++}_{ij}-\eta^{+-}_{ij} &= 2E_j\\
    \implies \quad \frac{1}{\eta^{++}_{ij}\eta^{+-}_{ij}} &= \frac{1}{2E_j}\left(\frac{1}{\eta^{+-}_{ij}} - \frac{1}{\eta^{++}_{ij}}\right)\\
    \implies \quad \frac{1}{\eta^{+-}_{ij}} &= \frac{E_j}{\eta^{++}_{ij}\eta^{+-}_{ij}} + \frac{1}{\eta^{++}_{ij}}
\end{align}
This result can now be applied to each term in \eqref{eq:ltd}
\begin{align*}
    I(q,p,m_\psi) =& \int d^3 \vec{k} \frac{1}{(2\pi)^3} \frac{1}{(2E_1)(2E_2)(2E_3)}\Bigg[\\
      &  \left(\frac{1}{\eta_{12}^{+-}}-\frac{1}{\eta_{12}^{++}}\right)\left(\frac{1}{\eta_{13}^{+-}}-\frac{1}{\eta_{13}^{++}}\right)\\
      &+ \left(\frac{1}{\eta_{21}^{+-}}-\frac{1}{\eta_{21}^{++}}\right)\left(\frac{1}{\eta_{23}^{+-}}-\frac{1}{\eta_{23}^{++}}\right)\\
      &+ \left(\frac{1}{\eta_{31}^{+-}}-\frac{1}{\eta_{31}^{++}}\right)\left(\frac{1}{\eta_{32}^{+-}}-\frac{1}{\eta_{32}^{++}}\right)
    \Bigg]
\end{align*}
We now have to factor out all the brackets.
\begin{align*}
I(q,p,m_\psi) =& \int d^3 \vec{k} \frac{1}{(2\pi)^3} \frac{1}{(2E_1)(2E_2)(2E_3)}\Bigg[\\
    &\frac{1}{\eta_{12}^{+-}\eta_{13}^{+-}} - \frac{1}{\eta_{12}^{+-}\eta_{13}^{++}} - \frac{1}{\eta_{12}^{++}\eta_{13}^{+-}} + \frac{1}{\eta_{12}^{++}\eta_{13}^{++}}\\
    &+ \frac{1}{\eta_{21}^{+-}\eta_{23}^{+-}} - \frac{1}{\eta_{21}^{+-}\eta_{23}^{++}} - \frac{1}{\eta_{21}^{++}\eta_{23}^{+-}} + \frac{1}{\eta_{21}^{++}\eta_{23}^{++}}\\
    &+ \frac{1}{\eta_{31}^{+-}\eta_{32}^{+-}} - \frac{1}{\eta_{31}^{+-}\eta_{32}^{++}} - \frac{1}{\eta_{31}^{++}\eta_{32}^{+-}} + \frac{1}{\eta_{31}^{++}\eta_{32}^{++}}
\Bigg]
\end{align*}
TODO
\begin{align*}
I(q,p,m_\psi) =& \int d^3 \vec{k} \frac{1}{(2\pi)^3} \frac{1}{(2E_1)(2E_2)(2E_3)}\Bigg[\\
      &\frac{1}{\eta_{21}^{++}\eta_{31}^{++}} + \frac{1}{\eta_{12}^{++}\eta_{13}^{++}}
    + \frac{1}{\eta_{12}^{++}\eta_{32}^{++}} + \frac{1}{\eta_{21}^{++}\eta_{23}^{++}}
    + \frac{1}{\eta_{13}^{++}\eta_{23}^{++}} + \frac{1}{\eta_{31}^{++}\eta_{32}^{++}}
\Bigg]
\end{align*}

