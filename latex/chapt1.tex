\section{Introduction}
%The numerical evaluation of multi-loop integrals in momentum space requires methods capable of efficiently regulating and isolating singular behavior. Threshold subtraction has proven to be an effective framework in this regard~\cite{kermanschah_numerical_2022} TODO ADD MORE. It was initially developed to address threshold singularities in~\cite{capatti_loop_2019} and subsequently extended to more intricate intersecting structures in~\cite{kermanschah_numerical_2022}.
%
%Existing applications, however, predominantly focus on integrals with real internal masses. In settings involving unstable particles, complex masses are unavoidable, and their presence modifies the location and character of the singularities. Whether the threshold subtraction method remains stable and reliable in this broader context is not a trivial question.
%
%In this work, I investigate the scalar triangle integral with complex internal masses as a controlled environment in which to test this extension. The triangle topology is sufficiently simple to allow for a transparent analysis, yet it retains essential features introduced by complex masses, including shifted branch cuts and modified threshold configurations. Demonstrating that threshold subtraction can be applied successfully in this setting broadens the scope of the method and represents a step toward its use in realistic multi-loop computations formulated within complex-mass schemes.

\section{Scalar Triangle Integral}
I consider the scalar Triangle integral, restricted to complex masses with $\Re(m_i^2) > 0$ and $\Im(m_i^2) \leq 0$.
\begin{figure}[H]
    \centering
     \feynmandiagram[horizontal=t2 to a] {
    % incoming particles on the left
    p1 [x=0, y=0]  -- [momentum =\(\vb{p}_1\)] t1 [x=3, y=0] , 
    p2 [x=0, y=-4] -- [momentum =\(\vb{p}_2\)] t3 [x=3, y=-4],

    % triangle loop
    t1 -- [momentum =\(\vb{k} + \vb{p}_1\), edge label' = \(m_1\)] t2 
    -- [momentum =\(\vb{k} - \vb{p}_2\), edge label' = \(m_3\)] t3 
    -- [momentum =\(\vb{k}\), edge label' = \(m_2\)] t1,

    % outgoing particle on the right
    t2[x = 6.5, y=-2] -- [momentum=\(\vb{p}_1+\vb{p}_2\)]a [x = 9.5, y = -2],
}; 
   \caption{Feynman diagram of the scalar Triangle integral}\label{fig:triangle_feynman}
\end{figure}
The Feynman diagram in Figure~\ref{fig:triangle_feynman} produces equation~\eqref{eq:triangle_integral}.
\begin{equation}
    I = \int \frac{\dd[4]{k}}{{(2\pi)}^4} \frac{i}{D_1 \, D_2 \, D_3} \label{eq:triangle_integral}
\end{equation}
with
\begin{equation}
    D_i = {(\vb{k}-\vb{q}_i)}^2-{\qty(m-i\varepsilon)}^2 = {(\vb{k}-\vb{q}_i)}^2-m_i^2 + i \varepsilon  \label{eq:denom}
\end{equation}
and
\begin{align}
    \vb{q}_1 = \vb{0}\\
    \vb{q}_2 = - \vb{p}_1 \\
    \vb{q}_3 = \vb{p}_2
\end{align}

\subsection{Integration of the Loop Energy}
There are many ways to perform the loop energy integration in eq.~\eqref{eq:triangle_integral}. Following the steps in~\cite{catani_loops_2008} to rederive the Loop Tree Duality (LTD) representation by integrating over the energy and then algebraically manipulating the resulting expression to get a result in the so called the Cross Free Family (CFF) representation derived in~\cite{capatti_exposing_2023}.

I will perform the $k^0$ integration with Cauchy's theorem. For this we need to consider the poles of the integrand from Eq.~\eqref{eq:triangle_integral} in $k^0$. These are given by
\begin{equation}
    k^0 = q_i^0 \pm \qty(\sqrt{{\qty(\va{k}-\va{q}_i)}^2+m_i^2} - i\varepsilon) = q_i^0 \pm E_i
\end{equation}
with
\begin{equation}
    E_i = \sqrt{{\qty(\va{k}-\va{q}_i)}^2+m_i^2} - i\varepsilon
\end{equation}
Closing the contour around the poles corresponding to the principal square roots leads to the contour shown in Figure~\ref{fig:poles}
\begin{figure}[H]
    \centering
    \begin{tikzpicture}[scale=1.2,>=latex]

% Axes
\draw[->,gray] (-3.5,0)--(3.5,0) node[right]{$\mathrm{Re}\,k^0$};
\draw[->,gray] (0,-3.5)--(0,1.5) node[above]{$\mathrm{Im}\,k^0$};

% Red poles (values roughly matching Python example)
\foreach \x/\y/\lbl in {
  -2/0.1/{-E1},
  0.5/-0.1/{+E1},
  -1.25/0.1/{-E2},
  1.25/-0.1/{+E2},
  -0.5/0.1/{-E3},
  2/-0.1/{+E3}
}{
  \fill[red] (\x,\y) circle(1.8pt);
  \node[above=3pt] at (\x,\y) {\lbl};
}

% Contour (clockwise semicircle in lower half-plane)
\draw[thick,blue,
  decoration={markings,mark=at position 0.25 with {\arrow{>}},
  mark=at position 0.75 with {\arrow{>}}},
  postaction={decorate}
  ] (3,0) arc[start angle=0,end angle=-180,radius=3];

% Lower line along real axis (closing contour)
\draw[thick,blue,postaction={decorate},
  decoration={markings,mark=at position 0.25 with {\arrow{>}},
  mark=at position 0.75 with {\arrow{>}}}
  ] (-3,0)--(3,0);

% Labels for contour
\node[blue,below right] at (2,-2.2) {$\Gamma$};
\node[below right] at (3,0) {$R$};
\node[below left] at (-3,0) {$-R$};

\end{tikzpicture}
    \caption{Poles of the denominator}\label{fig:poles}
\end{figure}

We can now apply Cauchy's theorem to perform the $k^0$ integral.
\begin{equation}
    I = \int \frac{\dd[3]{k}}{{(2\pi)}^4} 2\pi i \sum_{i= 1}^{3} \On{Res}_{k^0 = q^0_i + E_i}\qty[\frac{i}{D_1D_2D_3}] \label{eq:cauchy}
\end{equation}

Let's compute the residues.
\begin{equation}
    \On{Res}_{k^0 = q^0_i + E_i}\qty[\frac{i}{D_1D_2D_3}] = {\qty[\frac{i}{D_i'}\prod_{i \neq j} \frac{1}{D_j}]}_{k^0=q^0_i + E_i}= \frac{i}{2E_i} \prod_{i\neq j} \eval{\frac{1}{D_j}}_{k^0=q^0_i+E_i} \label{eq:residue}
\end{equation}

\newcommand{\ELIPT}[2]{\eta_{#1#2}^{++}}
\newcommand{\HYPER}[2]{\eta_{#1#2}^{+-}}

We can also introduce $\ELIPT{i}{j}$ called E-Surface and $\HYPER{i}{j}$ called H-Surface.

\begin{align}
    \eval{D_j}_{k^0=E_i} &= {\qty(q^0_i+E_i-q^0_j)}^2 - \overbrace{{\qty(\va{k}-\va{q}_j)}^2+m_j^2}^{E_j^2}\\
    &= E_i^2+{q_i^0}^2+{q_j^0}^2-2q_i^0q_j^0+2q_i^0E_i - 2q_j^0E_i-E_j^2\\
    &= \underbrace{\left(E_i+E_j+q_i^0-q_j^0\right)}_{\ELIPT{i}{j}}\underbrace{\left(E_i-E_j+q_i^0-q_j^0\right)}_{\HYPER{i}{j}} \label{eq:surfaces}
\end{align}
Inserting the results from Eq.~\eqref{eq:surfaces} and Eq.~\eqref{eq:residue} into Eq.~\eqref{eq:cauchy} we get the Loop Tree duality expression.
\newcommand{\BLTD}[3]{2E_{#1}\ELIPT{#1}{#2}\HYPER{#1}{#2}\ELIPT{#1}{#3}\HYPER{#1}{#3}}
\begin{equation}
    I =  \int \frac{\dd[3]{k}}{{(2\pi)}^3}\qty(\frac{1}{\BLTD{1}{2}{3}}+\frac{1}{\BLTD{2}{1}{3}}+\frac{1}{\BLTD{3}{1}{2}}) \label{eq:ltd}
\end{equation}

While this is a nice expression, it is not yet suited for numeric integration, since all of the H-Surfaces contain singularities which would make the procedure numerically unstable. It turns out however that these singularities are spurious and can be removed by purely algebraic manipulations, shown in detail in Appendix~\ref{sec:improved_ltd}. The resulting expression contains only E-Surfaces.

\begin{gather}
    I =  \int \dd[3]{k} \mathcal{I}_{CFF} \\ 
    \mathcal{I}_{CFF} = \frac{1}{{(4\pi)}^3} \frac{1}{E_1E_2E_3} \qty(
    \frac{1}{\ELIPT{2}{1}\ELIPT{3}{1}}
    + \frac{1}{\ELIPT{1}{2}\ELIPT{1}{3}}
    + \frac{1}{\ELIPT{1}{2}\ELIPT{3}{2}}
    + \frac{1}{\ELIPT{2}{1}\ELIPT{2}{3}}
    + \frac{1}{\ELIPT{1}{3}\ELIPT{2}{3}}
    + \frac{1}{\ELIPT{3}{1}\ELIPT{3}{2}}
    )
\label{eq:cff}
\end{gather}

\subsection{E-Surface}
The E-Surfaces themselves may still have singularities. We will now study them more closely.
Let us consider the problem in hemispherical coordinates centered around $\va{k_0}$. We also introduce $\va{k'}$ to simplify the algebra.
\begin{equation}
    \va{k} = k \, \hat{\vb{k}} + \va{k_0}
\end{equation}
We now introduce a shift to our coordinate system to simplify the following considerations.
\begin{align}
    \va{k'} &= \va{k} - \frac{1}{2}(\va{q}_i+\va{q}_j) \\
    &= k \, \hat{\vb{k}} + \underbrace{\va{k_0} - \frac{1}{2}(\va{q}_i+\va{q}_j)}_{\va{k'_0}}
\end{align}

Defining $\vb{q} = \frac{1}{2}\qty(\vb{q}_i-\vb{q}_j)$, the E-Surface simplifies to the following form.
\begin{equation}
    \ELIPT{i}{j} =  \sqrt{{(\va{k'}-\va{q})}^2+m_i^2} + \sqrt{{(\va{k'}+\va{q})}^2+m_j^2} + 2q^0 \overset{!}{=} 0
\end{equation}
\subsubsection{E-Surface existence condition}\label{sec:e_surface_exist}
Notice that if $q^0 > 0$ there exists no solution. If $q^0 < 0$ the square roots of the E-Surface can be eliminated by considering the following identity
\begin{align}
    \sqrt{A}+\sqrt{B} &= C\\
    \implies\quad 2\sqrt{AB} &= C^2-A-B\\
    \implies\quad 4AB &= {(C^2-A-B)}^2\\
    \implies\quad C^2 - 2(A+B) + {\qty(\frac{A-B}{C})}^2 &= 0
\end{align}
Applying this identity to the E-surface equation we find
\begin{align*}
    A-B &= -4\va{k'}\cdot \va{q} + m_i^2 - m_j^2\\
    A+B &= 2\va{k'}^2 + 2 \va{q}^2 + m_i^2 + m_j^2\\
\end{align*}
\begin{equation} 
    4{q^0}^2 - 2 \qty(2\va{k'}^2 + 2 \va{q}^2 + m_i^2 + m_j^2) + {\qty(\frac{-4 \va{k'}\cdot \va{q} + m_i^2 - m_j^2}{q^0})}^2= 0
\end{equation}
Which can be simplified into a quadratic equation in $\va{k'}$
\begin{equation}
    \vb{q}^2-\frac{m_i^2 + m_j^2}{2} - \va{k'}^2 + {\qty(\va{k'}\cdot \frac{\va{q}}{q^0})}^2 - \frac{m_i^2 - m_j^2}{4q^0} \qty(\va{k'}\cdot \frac{\va{q}}{q^0}) + {\qty(\frac{m_i^2 - m_j^2}{4q^0})}^2 = 0
\end{equation}
\begin{equation}
    \va{v} = \frac{\va{q}}{q^0}
\qquad
    \braket{m^2} = \frac{m_i^2 + m_j^2}{2}
\qquad
    \Delta = \frac{m_i^2 - m_j^2}{4q^0}
\end{equation}
With these definitions we can rewrite the quadratic equation as
\begin{equation}
    \va{k'}^2 - {\qty(\va{k'}\cdot\va{v})}^2 + \Delta \qty(\va{k'}\cdot\va{v}) = \vb{q}^2-\braket{m^2} + \Delta^2
\end{equation}
By inserting the parametrization $k \, \hat{\vb{k}} + \va{k'_0}$ we finally get a quadratic equation in the radial variable $k$.
\begin{equation}
    \alpha k^2 + \beta k + \gamma - i\varepsilon = 0
\end{equation}
with
\begin{align}
    \alpha &= 1-{\qty(\hat{\vb{k}}\cdot\va{v})}^2\\
    \beta &= 2(\hat{\vb{k}}\cdot\va{k'_0})-2(\hat{\vb{k}}\cdot\va{v})(\va{k'_0}\cdot\va{v})+\Delta(\hat{\vb{k}}\cdot\va{v})\\
    \gamma &= \va{k'_0}^2-{\qty(\va{k'_0}\cdot\va{v})}^2+\Delta(\va{k'_0}\cdot\va{v})-\vb{q}^2-\braket{m^2}+\Delta^2
\end{align}



In the special case where $m_i = m_j = m$ we can also make the following simplifications.
\begin{equation}
    {\va{k'}}^2 - {\qty(\va{k'}\cdot\va{v})}^2 = \vb{q}^2-m^2 \qquad \text{with} \qquad \va{v} = \frac{\va{q}}{q^0} \label{eq:implicit_e_surf}
\end{equation}
Using the hemispherical parametrization, we express the loop momentum as
\begin{equation}
    \va{k'} = k \vb{\hat{k}} + \va{k'_0}
\end{equation}
By inserting this into equation~\eqref{eq:implicit_e_surf} we find an equation that is quadratic in the radial variable $k$.
\begin{equation}
    {\qty(k \, \hat{\vb{k}} + \va{k'_0})}^2 - {\qty((k \, \hat{\vb{k}} + \va{k'_0})\cdot \va{v})}^2 = \vb{q}^2-m^2
\end{equation}
At this point we will reintroduce the causal prescription to make the branch choice explicit. Considering the introduction of the causal prescription in equation~\eqref{eq:denom}, we can see that this can be done by replacing $m^2 \to m^2 - i \varepsilon$.

Reintroducing the causal prescription here is essential for a simple reason:
without the $-i \varepsilon$, the discriminant of the quadratic may lie exactly on a branch cut or coincide with a real threshold, and the standard quadratic formula would give ambiguous or discontinuous solutions.

With the prescription restored we can bundle the terms into the quadratic equation~\eqref{eq:e_surface_quad}. The two roots $k^*_-$ and $k^*_+$ of this equation correspond precisely to the correct E-surface intersections.
\begin{equation}
    \qty[1-\hat{\vb{k}} \cdot \va{v}] k^2 +  2 \qty[\hat{\vb{k}} \cdot \va{k'_0} - \qty(\hat{\vb{k}} \cdot \va{v})\qty(\va{k'_0} \cdot \va{v})]\, k + \qty[\va{k_0}^2 - {\qty(\va{k'_0} \cdot \va{v})} - \vb{q}^2 + m^2 - i\varepsilon] = 0 \label{eq:e_surface_quad}
\end{equation}
The geometry of the E-surface is made even more transparent when choosing the parametrization center such that $\va{k'_0} = \va{0}$. In this case the expression above can simplify to the following well-known parametrization of an ellipsoid.
\begin{equation}
    k^2 = \frac{\vb{q}^2-m^2}{1-{\qty(\hat{\vb{k}} \cdot \va{v})}^2}
\end{equation}
\section{Threshold subtraction}\label{sec:thresh}
Threshold singularities appear whenever the loop momentum intersects an E-surface. The idea of threshold subtraction is straightforward: isolate the pole generated by a single E-surface and subtract a locally defined counterterm that reproduces its behavior exactly at the singular point. This ensures that the combined integrand remains finite everywhere in the integration domain, while the counterterm itself can be integrated analytically or via a dedicated parametrization. The entire construction is local, depends only on geometric data of the corresponding ellipsoid, and generalizes cleanly when multiple E-surfaces overlap~\cite{}.
\subsection{Subtraction of a single E-Surface}
To remove the threshold singularity corresponding to an E-surface, we will define a local counterterm for an integrand of the form
\begin{equation}
    \mathcal{I} = \frac{f(\va{k})}{\ELIPT{i}{j}}
\end{equation}
where the integration space is parametrized hemispherically with the center $\va{k}_0$ chosen inside of the ellipsoid.
This choice is necessary to ensure local cancellations~\cite{}.
The zeros of $\ELIPT{i}{j}$ along the radial direction can then be found with Eq.~\eqref{eq:e_surface_quad}.
There are always two solution, which we will denote $k^*_+$ and $k^*_-$.

The singular behavior of the integrand can fully be captured by the first-order taylor expansion in $\ELIPT{i}{j}$ around the pole. Let us consider only the pole $k^*_{\pm}$ for now.
\begin{equation}
    \ELIPT{i}{j} = 0 + (k-k^*_{\pm}) \underbrace{\eval{\pdv{k} \ELIPT{i}{j}}_{\va{k}=\va{k^*_{\pm}}}}_{\eta'} + \order{k^2}
\end{equation}
We can now define a counterterm that will cancel the singularity in $k^*_{\pm}$.
\begin{equation}
    \on{CT}_{ij}(k^*_{\pm}) = \chi (\va{k}) \frac{f(\va{k^*_{\pm}})}{(k-k^*_{\pm}) \, \eta'}
\end{equation}
where $\chi (\va{k})$ is an arbitrary mask, that is smooth at $\chi (\va{k}^*_{\pm}) = 1$ and vanishing as $\abs{k-k^*_{\pm}} \to \infty$
The partial derivative along the radial variable $k$ is straightforward to compute explicitly.
\begin{equation}
    \eval{\pdv{k} \ELIPT{i}{j}}_{\va{k}=k^*_{\pm} \hat{\vb{k}} } = {\qty[\frac{1}{E_i}(\va{k}-\va{q}_i)\cdot \hat{\vb{k}} + \frac{1}{E_j}(\va{k}-\va{q}_j)\cdot \hat{\vb{k}} ]}_{\va{k} = k^*_{\pm} \hat{\vb{k}}}
\end{equation}

By construction, subtracting the counterterm $\on{CT}_{ij}(k^*_{\pm})$ will remove the singularity at $k^*_{\pm}$ from the integrand. By choosing the mask $\chi$ appropriately we can ensure that it is possible integrate along the radius of the counterterm analytically. This allows to add the subtracted counterterm back to the final result in an integral in a way that completely removes the singularity.
\subsection{Radially integrating the counterterm}
The choice of the mask $\chi$ is crucial to enable analytical radial integration of the counterterm.
Writing the counterterm integral in hemispherical coordinates centered at $\va{k_0}$, we have

\begin{equation}
    I_{\on{CT}} = \frac{1}{2} \int_{S^2} \dd[2]{\hat{\vb{k}} } \int_{0}^{\infty} \dd{k} {k}^2 \on{CT}(\vec{k})
\end{equation}

Several useful observations can be made on the choice of $\chi$.
\begin{itemize}
    \item By choosing $\chi = \frac{{k^*\pm}^2}{k^2}$, the Jacobian factor $k^2$ is canceled, leaving a simple $\frac{1}{k-k^*_{\pm}}$ integrand for the radial integral.
    \item If $\chi = \frac{{k^*_{\pm}}^2}{k^2} f(k)$ where $f$ is an even function, the real part of the integration vanishes, simplifying the computation.
\end{itemize}
Care must be taken when canceling the Jacobian factor this way. The parametrization of the subtracted integrand must match the parametrization of the counterterm, or other additional steps must be taken, to not introduce additional singularities into the integrand.

For our purposes we choose a mask that is only nonzero in a finite region.
\begin{equation}
    \chi{\va{k}} = \begin{cases}
        \frac{{k^*_{\pm}}^2}{k^2} & k \in (\lambda_1, \lambda_2)\\
        0& \text{else}
    \end{cases}
\end{equation}
The radial integration is now straightforward.
\begin{equation}
   \int \dd{k} I_{\on{CT}}(k^*_{\pm}) = \int_{\lambda_1}^{\lambda_2} \dd{k} \frac{1}{k-k^*_{\pm}} = \ln(\frac{\lambda_2-k^*_{\pm}}{\lambda_1-k^*_{\pm}})
\end{equation}
In the special case, where $\lambda_1 = \Re k^*_{\pm} - \lambda$ and $\lambda_2 = \Re k_{\pm}^* + \lambda$, the real part vanishes due to symmetry around $\Re k^*_{\pm}$, leaving
\begin{equation}
    \int \dd{k} =  -i \qty(\pi + 2\arctan(\frac{\Im k^*_{\pm}}{\lambda}))
\end{equation}
In this result we recover the original result TODO as $\Im k^*_{\pm} \to 0$.
Finally, including the angular integration and both $k^*_-$ and $k^*_+$, the full counterterm contribution of a single E-surface reads
\begin{equation}
    I_{\on{CT}} = \sum_{\pm} \frac{1}{2} \int_{S^2} \dd[2]{\hat{\vb{k}} } f(k^*) {k^*}^2 \ln(\frac{\lambda_2-k^*}{\lambda_1-k^*})
\end{equation}
The factor $\frac{1}{2}$ accounts for the fact that $\int_{S^2}$ integrates over the whole sphere thus integrating over two opposing hemispheres.
\subsection{Threshold subtraction in the actual integrand}
In my examples we will only ever choose parametrizations centered around the origin $k^0 = \va{0}$. Without loss of generality we can reorder $\vb{q}_i$ such that $\vb{q}_1 < \vb{q}_2 < \vb{q}_3$. Thus the existence condition derived in Section~\ref{sec:e_surface_exist} is fulfilled for the following E-surfaces:
\begin{itemize}
    \item $\ELIPT{1}{2}$
    \item $\ELIPT{2}{3}$
    \item $\ELIPT{1}{3}$
\end{itemize}
Each of these requires it's own counterterm.
As an example, consider $\ELIPT{1}{2}$.
First we need to collect all of the terms in the integrand containing the relevant E-surface $\ELIPT{1}{2}$. 
This factor corresponds exactly to the function $f$ introduced in Section~\ref{sec:thresh}.
\begin{equation}
    \ELIPT{1}{2} \qty{1}{{(4\pi)}^3}\frac{1}{E_1E_2E_3} \qty(\frac{1}{\ELIPT{1}{3}}+\frac{1}{\ELIPT{2}{3}}) = \ELIPT{3}{2} \, f(\va{k})
\end{equation}
The counterterm removing the pole at either $k^*_+$ or $k^*_-$ is therefore given by
\begin{equation}
    \on{CT}_{12}(k^*_\pm) = \chi(k-k^*) \, f(\va{k}) \,\frac{1}{\eta'_{12}} \,  \frac{1}{k-k^*_\pm}
\end{equation}
where $\eta_{12}'$ is the radial derivative of $\ELIPT{1}{2}$ evaluated at the pole.
The same procedure applies to $\ELIPT{2}{3}$ and $\ELIPT{1}{3}$
The integral is then finally given by
\begin{equation}
    I = \int \dd[3]{k} \qty(\mathcal{I}_{CFF} - \sum_{\pm} \sum_{(i,j) \in \{(1,2),(2,3),(1,3)\}} \on{CT}_{ij}(k^*_\pm)) + \sum_{\pm} \sum_{(i,j) \in \{(1,2),(2,3),(1,3)\}} I_{\on{CT}_{ij}}(k^*_\pm)
\end{equation}

\section{Numerical Integration}
The numerical integration is performed with the Python API of the \texttt{symbolica} library~\cite{symbolica}.
\texttt{Symbolica} supports compilation of symbolic expressions into optimized assembly code for fast evaluation. It also provides an efficient implementation of the \texttt{VEGAS} Monte Carlo integration algorithm, which is well suited for this problem.
\subsection{Parametrization}
\texttt{Symbolica} provides efficient sampling in $N$-dimensional hypercubes.
There are many reasonable choices to map from the unit hypercube to hemispherical coordinates. I choose the following mapping from unit hypercubes to the relevant parametrizations with their respective Jacobians $J$.

\subsubsection{parametrization of the unit hemisphere}
\begin{align}
    \vb{\hat{k}}(u, v) &= 
    \begin{pmatrix}
        \sin(\phi) \cos(2 \pi w)\\
        \sin(\phi) \sin(2 \pi w)\\
        \cos(\phi)
    \end{pmatrix}\\
    r &= \frac{u}{1-u} \\
    \cos(\phi) &= 1 - 2 v\\
    \sin(\phi) &= \sqrt{1-\cos^2(\phi)}\\
    J_{\vb{\hat{k}}} &= 2 \pi
\end{align}

\subsubsection{Spherical Parametrization}
\begin{align}
    \va{k}(u, v, w) &= r \vb{\hat{k}}(u, v) \\
    \tilde{u} &= 2\,u-1 \\
    r &= \tilde{u} - \frac{1}{\tilde{u}} \\
    J &= \frac{r^2 \, J_{\vb{\hat{k}}}}{{(1-\tilde{u})}^2}
\end{align}

\section{Results}

Plots for threshold subtraction.

Plots of numerical stability with relative error (dimensionless)

\section{Discussion}