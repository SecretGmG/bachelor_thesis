\chapter{Triangle Integral}

The goal is to prepare the scalar triangle integral $I$ for numerical
integration. e do this by integrating over the $k^0$ component analytically,
giving us the Loop Tree Duality (LTD) expression. This will then be further
simplified.

\begin{figure}[H]
    \centering
     \feynmandiagram [horizontal=t2 to a, node distance=2.5cm] {
    % incoming particles on the left
    p1 -- [momentum =\(\vb{p}_1\)] t1, p2 -- [momentum =\(\vb{p}_2\)] t3,

    % triangle loop
    t1 -- [momentum =\(\vb{k} + \vb{p}_1\)] t2 -- [momentum =\(\vb{k} - \vb{p}_2\)] t3 -- [momentum =\(\vb{k}\)] t1,

    % outgoing particle on the right
    t2 -- [momentum=\(\vb{p}_1+\vb{p}_2\)]a,

    % invisible edge keeps p1, p2 parallel
    p1 -- [opacity=0] p2 }; 
   \caption{Feynman diagram of the Triangle loop integral} \label{fig:triangle_feynman}
\end{figure}

\begin{equation}
    I = \int \frac{\dd[4]{k}}{{(2\pi)}^4} \frac{i}{D_1D_2D_3} \label{eq:triangle_integral}
\end{equation}
with
\begin{equation}
    D_i = {(\vb{k}-\vb{q}_i)}^2-m^2+i\varepsilon \label{eq:denom}
\end{equation}.
and
\begin{align}
    \vb{q}_1 = \vb{k}_0            \\
    \vb{q}_2 = \vb{k}_0 - \vb{p}_1 \\
    \vb{q}_3 = \vb{k}_0 + \vb{p}_2
\end{align}
here $\vb{k}_0$ can be chosen freely, due to integrating over the entire spacetime.

\section{Loop Tree Duality}
We will perform the $k^0$ integration with the cauchy theorem. hen closing the
contour such that only the poles coming from the principle square root are
included, and closing it ith a semicircle ith radius $R$ it is easy to see that
it's contribution vanishes as $R \to \infty$.
%\begin{equation}
%    \int_{0}^{-\pi} \dd{\phi} \frac{ir}{D_1D_2D_3} \sim \int_{0}^{-\pi} \dd{\phi} \order{r^{-5}}
%\end{equation}

The poles of the integrand from Eq.~\eqref{eq:triangle_integral} in $k^0$ are found to
be
\begin{equation}
    k^0 = q_i^0 \pm \qty(\sqrt{{\qty(\va{k}-\va{q}_i)}^2+m^2}-i\varepsilon) = q_i^0 \pm \qty(E_i - i\varepsilon)
\end{equation}
Notice that the principle square root preserves the sign of the imaginary part of the argument. Thus we need to close the contour below if $\Im{m^2}\leq0$ and above otherwise. From now on we will assume that $\Im{m^2}\leq$. This contour is shown in Figure~\ref{fig:poles}.
\begin{figure}[H]
    \centering
    \begin{tikzpicture}[scale=1.2,>=latex]

% Axes
\draw[->,gray] (-3.5,0)--(3.5,0) node[right]{$\mathrm{Re}\,k^0$};
\draw[->,gray] (0,-3.5)--(0,1.5) node[above]{$\mathrm{Im}\,k^0$};

% Red poles (values roughly matching Python example)
\foreach \x/\y/\lbl in {
  -2/0.1/{-E1},
  0.5/-0.1/{+E1},
  -1.25/0.1/{-E2},
  1.25/-0.1/{+E2},
  -0.5/0.1/{-E3},
  2/-0.1/{+E3}
}{
  \fill[red] (\x,\y) circle(1.8pt);
  \node[above=3pt] at (\x,\y) {\lbl};
}

% Contour (clockwise semicircle in lower half-plane)
\draw[thick,blue,
  decoration={markings,mark=at position 0.25 with {\arrow{>}},
  mark=at position 0.75 with {\arrow{>}}},
  postaction={decorate}
  ] (3,0) arc[start angle=0,end angle=-180,radius=3];

% Lower line along real axis (closing contour)
\draw[thick,blue,postaction={decorate},
  decoration={markings,mark=at position 0.25 with {\arrow{>}},
  mark=at position 0.75 with {\arrow{>}}}
  ] (-3,0)--(3,0);

% Labels for contour
\node[blue,below right] at (2,-2.2) {$\Gamma$};
\node[below right] at (3,0) {$R$};
\node[below left] at (-3,0) {$-R$};

\end{tikzpicture}
    \caption{Poles of the denominator}
    \label{fig:poles}
\end{figure}

We can now apply cauchy's theorem to perform the $k^0$ integral.
\begin{equation}
    I = \int \frac{\dd[3]{k}}{{(2\pi)}^4} 2\pi i \sum_{i= 1}^{3} \On{Res}_{k^0 = q^0_i + E_i}\qty[\frac{i}{D_1D_2D_3}] \label{eq:cauchy}
\end{equation}

Let's compute the residues.
\begin{equation}
    \On{Res}_{k^0 = q^0_i + E_i}\qty[\frac{i}{D_1D_2D_3}] = \qty[\frac{i}{D_i'}\prod_{\neq j} \frac{1}{D_j}]_{k^0=E_i}= \frac{i}{2E_i} \prod_{i\neq j} \eval{\frac{1}{D_j}}_{k^0=E_i} \label{eq:residue}
\end{equation}

\newcommand{\ELIPT}[2]{\eta_{#1#2}^{++}}
\newcommand{\HYPER}[2]{\eta_{#1#2}^{+-}}

We can also introduce $\ELIPT{i}{j}$ called E-Surface and $\HYPER{i}{j}$ called H-Surface to simplify the algebra.

\begin{align}
    \eval{D_j}_{k^0=E_i} &= {\qty(q^0_i+E_i-q^0_j)}^2 - \overbrace{{\qty(\va{k}-\va{q}_j)}^2+m^2}^{E_j^2}\\
    &= E_i^2+{q_i^0}^2+{q_j^0}^2-2q_i^0q_j^0+2q_i^0E_i - 2q_j^0E_i-E_j^2\\
    &= \underbrace{\left(E_i+E_j+q_i^0-q_j^0\right)}_{\ELIPT{i}{j}}\underbrace{\left(E_i-E_j+q_i^0-q_j^0\right)}_{\HYPER{i}{j}} \label{eq:surfaces}
\end{align}
Inserting the results from Eq.~\eqref{eq:surfaces} and Eq.~\eqref{eq:residue} into Eq.~\eqref{eq:cauchy} we get the Loop Tree duality expression.
\newcommand{\BLTD}[3]{2E_{#1}\ELIPT{#1}{#2}\HYPER{#1}{#2}\ELIPT{#1}{#3}\HYPER{#1}{#3}}
\begin{equation}
    I =  \int \frac{\dd[3]{k}}{{(2\pi)}^3}\qty(\frac{1}{\BLTD{1}{2}{3}}+\frac{1}{\BLTD{2}{1}{3}}+\frac{1}{\BLTD{3}{1}{2}}) \label{eq:naive_ltd}
\end{equation}

While this is a nice expression, it is not yet suited for numeric integration, since all of the H-Surfaces contain singularities which would make the procedure numerically unstable. It turns out however that these singularities are spurious and can be removed by purely algebraic manipulations, shown in detail in Appendix~\ref{sec:improved_ltd}. The resulting expression contains only E-Surfaces.

TODO: MAYBE MENTION CFF HERE!

\begin{equation}
    I
    =  \int \frac{\dd[3]{k}}{{(4\pi)}^3} \frac{1}{E_1E_2E_3} \qty(
    \frac{1}{\ELIPT{2}{1}\ELIPT{3}{1}}
    + \frac{1}{\ELIPT{1}{2}\ELIPT{1}{3}}
    + \frac{1}{\ELIPT{1}{2}\ELIPT{3}{2}}
    + \frac{1}{\ELIPT{2}{1}\ELIPT{2}{3}}
    + \frac{1}{\ELIPT{1}{3}\ELIPT{2}{3}}
    + \frac{1}{\ELIPT{3}{1}\ELIPT{3}{2}}
    )
\label{eq:improved_ltd}
\end{equation}

\section{E-Surfaces}
The E-Surfaces themselves may still have singularities. We will now study them more closely.
Let us consider the problem ins spherical coordinates
\begin{equation}
\va{k} = \va{k'} + \frac{1}{2}(\va{q}_i+\va{q}_j) \qquad \text{with} \qquad \va{k'} = k \, \hat{k} + \va{k}_0
\end{equation}
Defining $\vb{q} = \frac{1}{2}\qty(\vb{q}_i-\vb{q}_j)$, the E-Surface simplifies to the following form.
\begin{equation}
    \ELIPT{i}{j} =  \sqrt{{(\va{k'}-\va{q})}^2+m^2} + \sqrt{{(\va{k'}+\va{q})}^2+m^2} + 2q^0 \overset{!}{=} 0
\end{equation}
Notice that the square roots of the E-Surface can be eliminated:
\begin{align}
    \sqrt{A}+\sqrt{B} &= C\\
    2\sqrt{AB} &= C^2-A-B\\
    4AB &= {(C^2-A-B)}^2
\end{align}
When we now insert and simplify, we get the following expression, which is a second order polynomial.
\begin{equation}
    {(\va{k'}\cdot\va{q})}^2-({\va{k'}}^2+\va{q}^2+m^2){q^0}^2+{q^0}^4 = 0
\end{equation}

The geometry of the solution is more apparent when we insert the parametrization.
\begin{align}
    %(\va{k}\cdot\va{q})^2-(\va{k}^2+\va{q}^2+m^2){q^0}^2+{q^0}^4 = 0\\
    %&= k^2 \qty(\hat{k}\cdot \va{q})^2 - 2 k (\hat{k} \cdot \va{q})(\va{k}_0 \cdot \va{q}) + (\va{k}_0\cdot \va{q})^2
    %-k^2 {q^0}^2 + 2 k \hat{k}\cdot \va{k}_0 {q^0}^2 + \qty({q^0}^2 - \va{q}^2 - m^2) {q^0}^2\\
    k^2 \qty(1-{\qty(\frac{\hat{k}\cdot \va{q}}{q^0})}^2) + 2 k \hat{k}\cdot \va{k}_0 - \qty(\vb{q}^2 - m^2) = 0 \label{eq:e_surface_quad}
\end{align}
In the case where $\va{k}_0 = \va{0}$ it is made explicit, that the E-Surface is an ellipsoid and radially symmetric around the axis $\va{q}$.
\begin{equation}
    {k}^2 = (\vb{q}^2-m^2){\qty(1-{\qty(\frac{\hat{k}\cdot\va{q}}{q^0})}^2)}^{-1}
\end{equation}

In this form it is also more convenient to find an existence condition for the E-Surface.
First notice that, because the external momenta are all timelike, $\va{q}$ is also timelike. 
This implies 
\begin{equation}
    {\qty(1-{\qty(\frac{\va{\sigma}\cdot\va{q}}{q^0})}^2)}^{-1} > 0
\end{equation}
We can now easily see that there only exist positive solutions for $k$ if $\vb{q}^2>m^2$

\section{Threshold subtraction}
EXPLAIN BASIC IDEA OF TRHRESHOLD SUBTRACTION

\subsection{Subtraction of a single E-Surface}
It makes sense to chose a parametrization in which there always exists a dimension where the singularity is integrable.
The approach we will use is to parameterize the space spherically, choosing the centre inside the ellipsoid.
The zeros along the radius can then be found with Eq.~\eqref{eq:e_surface_quad}.
Due to this choice there will always only exist one positive solution which we will call $k^*$

IDEA:
CHOOSE CENTER EXACTLY IN MIDDLE, THEN PARAMETERIZE SO THAT VALUES ARE LOCAL ONLY NEAR CROSSOVER IN A WAY THAT ALWAYS WORKS.
THIS SHOULD THEN ALSO WORK FOR THE DEGENERATE CASE WHERE THE SURFACES TOUCH??
MIGHT BE MORE ELEGANT. BUT THE NECEASSARY TRANSFORMATION COULD BE COMPLICATED

TODO

Let us Taylor expand $\ELIPT{i}{j}$ to the first order in $k$ around the zero $k^*$, also define $\va{k}^* = k^*\,\hat{k}+\va{k}_0$
\begin{equation}
    \ELIPT{i}{j} = 0 + (k-k^*) \eval{\pdv{k} \ELIPT{i}{j}}_{\va{k}=\va{k}^*} + \order{k^2}
\end{equation}

Remember that:
\begin{align}
    I &= \int \frac{\dd[3]{k}}{{(2\pi)}^3} \frac{1}{(2E_1)(2E_2)(2E_3)} \qty(\frac{1}{\ELIPT{2}{1}\ELIPT{3}{1}} + \dots)\\
    &= \int_{0}^{\infty} \dd{k} \int_{S^2} \dd[2]{\sigma} k^2 \qty(\frac{1}{(2E_1)(2E_2)(2E_3)} \frac{1}{\ELIPT{2}{1}\ELIPT{3}{1}} + \dots)
\end{align}

We can now define the form of our counterterm that will cancel the singularity in $k^*$.
\begin{equation}
    \on{CT} = \frac{\chi (\va{k})}{(k-k^*)} \frac{\eval{f}_{\va{k}^*}}{f} N(\hat{k})
\end{equation}
Where $f\neq 0$ is an arbitrary smooth function, $\chi (\va{k})$ is a mask, that is smooth at $\chi (\va{k}^*) = 1$ and $N(\va{\sigma})$ is a constant defined below.
\begin{equation}
    N(\hat{k}) = {\qty[\frac{1}{\pdv{k} \ELIPT{i}{j}}\frac{1}{(2E_1)(2E_2)(2E_3)} \frac{1}{\ELIPT{k}{l}}]}_{\va{k}^*}
\end{equation}
The partial derivative is easy to compute
\begin{equation}
    \eval{\pdv{k} \ELIPT{i}{j}}_{\va{k}=k \hat{k}} = {\qty[\frac{1}{E_i}(\va{k}-\va{q}_i)\cdot \hat{k} + \frac{1}{E_j}(\va{k}-\va{q}_j)\cdot \hat{k}]}_{\va{k} = \va{k}^*}
\end{equation}

TODO: show that the limit $\lim_{k\to k^*} \frac{1}{\ELIPT{i}{j}} - \on{CT}$ is actually finite!

Parameterizing the space spherically centered around $\va{k}_0$
\begin{equation}
    I_{\on{CT}} = \int_{S^2} \dd[2]{\hat{k}} \int_{0}^{\infty} \dd{k} {k}^2 \on{CT}(\vec{k})
\end{equation}
Here we choose $f(\va{k}) = k^2$ such that the integration measure $k^2$ will be canceled out exactly.
If we also choose $\chi(\va{k}) = \Theta(\lambda - \abs{k^*-k})$ with $\lambda < k^*$ we get a particularly simple integral.
\begin{equation}
    \chi{\va{k}} = \begin{cases}
        1&k \in (\lambda_1, \lambda_2)\\
        0& \text{else}
    \end{cases}
\end{equation}
The radial integration is now easy to compute
\begin{equation}
    I_{CT, R} = \int_{-\lambda}^{\lambda} \dd{k} \frac{1}{k-k^*} = \ln(\frac{\lambda_1-k^*}{\lambda_2-k^*})
\end{equation}
Let's consider the special case where $\lambda_1 = \Re k^* - \lambda$ and $\lambda_2 = \Re k^* + \lambda$.
Then the real part of this integration is always zero because $\abs{\lambda-i\Im k^*} = \abs{-\lambda-i\Im k^*}$, we find.
\begin{equation}
    I_{CT, R} =  -i\qty(\pi + 2\atan(\frac{\Im k^*}{\lambda}))
\end{equation}
Notice that this result implies that the counter term integration does not depend on the causal prescription!
\begin{equation}
    I_{CT} = -i\int_{S^2} \dd[2]{\hat{k}} N(\hat{k}) I_{CT, R}
\end{equation}
The calculations above hold for arbitrary values of $k^* \not\in (-\infty,0]$ % chktex 9 chktex 17

\subsubsection{Threshold subtraction in the actual integrand}
Local cancelation of singularities. That's why we need to parameterize such that the parametrization match on the cross section.
To keep both radii integrable in the way described above, the center needs to be chosen inside both surfaces.

what if they touch? still possible to cancel locally by aligning. prob not necessary, but this might also increase stability if the almost touch.
 

