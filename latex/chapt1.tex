\chapter{Triangle Integral}

The goal is to prepare the scalar triangle integral $I$ for numerical
integration. e do this by integrating over the $k^0$ component analytically,
giving us the Loop Tree Duality (LTD) expression. This ill then be further
simplified (MENTION CROSS FREE FAMILY).

\begin{figure}[H]
    \centering
     \feynmandiagram [horizontal=t2 to a, node distance=2.5cm] {
    % incoming particles on the left
    p1 -- [momentum =\(\vb{p}_1\)] t1, p2 -- [momentum =\(\vb{p}_2\)] t3,

    % triangle loop
    t1 -- [momentum =\(\vb{k} + \vb{p}_1\)] t2 -- [momentum =\(\vb{k} - \vb{p}_2\)] t3 -- [momentum =\(\vb{k}\)] t1,

    % outgoing particle on the right
    t2 -- [momentum=\(\vb{p}_1+\vb{p}_2\)]a,

    % invisible edge keeps p1, p2 parallel
    p1 -- [opacity=0] p2 }; 
   \caption{Feynman diagram of the Triangle loop integral} \label{fig:triangle_feynman}
\end{figure}

\begin{equation}
    I = \int \frac{\dd[4]{k}}{{(2\pi)}^4} \frac{i}{D_1D_2D_3} \label{eq:triangle_integral}
\end{equation}
with
\begin{equation}
    D_i = {(\vb{k}-\vb{q}_i)}^2-m^2+i\varepsilon \label{eq:denom}
\end{equation}.
and
\begin{align}
    \vb{q}_1 = \vb{k}_0            \\
    \vb{q}_2 = \vb{k}_0 - \vb{p}_1 \\
    \vb{q}_3 = \vb{k}_0 + \vb{p}_2
\end{align}
here $\vb{k}_0$ can be chosen freely, due to integrating over the entire spacetime.

\section{Loop Tree Duality}
We will perform the $k^0$ integration with the cauchy theorem. hen closing the
contour such that only the poles coming from the principle square root are
included, and closing it ith a semicircle ith radius $R$ it is easy to see that
it's contribution vanishes as $R \to \infty$.
%\begin{equation}
%    \int_{0}^{-\pi} \dd{\phi} \frac{ir}{D_1D_2D_3} \sim \int_{0}^{-\pi} \dd{\phi} \order{r^{-5}}
%\end{equation}

The poles of the integrand from Eq.~\eqref{eq:integral} in $k^0$ are found to
be
\begin{equation}
    k^0 = q_i^0 \pm \qty(\sqrt{{\qty(\va{k}-\va{q}_i)}^2+m^2}-i\varepsilon) = q_i^0 \pm \qty(E_i - i\varepsilon)
\end{equation}
Notice that the principle square root preserves the sign of the imaginary part of the argument. Thus we need to close the contour below if $\Im{m^2}\leq0$ and above otherwise. From now on we will assume that $\Im{m^2}\leq$. This contour is shown in Figure~\ref{fig:poles}.
\begin{figure}[H]
    \centering
    \begin{tikzpicture}[scale=1.2,>=latex]

% Axes
\draw[->,gray] (-3.5,0)--(3.5,0) node[right]{$\mathrm{Re}\,k^0$};
\draw[->,gray] (0,-3.5)--(0,1.5) node[above]{$\mathrm{Im}\,k^0$};

% Red poles (values roughly matching Python example)
\foreach \x/\y/\lbl in {
  -2/0.1/{-E1},
  0.5/-0.1/{+E1},
  -1.25/0.1/{-E2},
  1.25/-0.1/{+E2},
  -0.5/0.1/{-E3},
  2/-0.1/{+E3}
}{
  \fill[red] (\x,\y) circle(1.8pt);
  \node[above=3pt] at (\x,\y) {\lbl};
}

% Contour (clockwise semicircle in lower half-plane)
\draw[thick,blue,
  decoration={markings,mark=at position 0.25 with {\arrow{>}},
  mark=at position 0.75 with {\arrow{>}}},
  postaction={decorate}
  ] (3,0) arc[start angle=0,end angle=-180,radius=3];

% Lower line along real axis (closing contour)
\draw[thick,blue,postaction={decorate},
  decoration={markings,mark=at position 0.25 with {\arrow{>}},
  mark=at position 0.75 with {\arrow{>}}}
  ] (-3,0)--(3,0);

% Labels for contour
\node[blue,below right] at (2,-2.2) {$\Gamma$};
\node[below right] at (3,0) {$R$};
\node[below left] at (-3,0) {$-R$};

\end{tikzpicture}
    \caption{Poles of the denominator}
    \label{fig:poles}
\end{figure}

We can now apply cauchy's theorem to perform the $k^0$ integral.
\begin{equation}
    I = \int \frac{\dd[4]{k}}{{(2\pi)}^4} 2\pi i \sum_{i= 1}^{3} \On{Res}_{k^0 = q^0_i + E_i}\qty[\frac{i}{D_1D_2D_3}] \label{eq:cauchy}
\end{equation}

Let's compute the residues.
\begin{equation}
    \On{Res}_{k^0 = q^0_i + E_i}\qty[\frac{i}{D_1D_2D_3}] = \qty[\frac{i}{D_i'}\prod_{\neq j} \frac{1}{D_j}]_{k^0=E_i}= \frac{i}{2E_i} \prod_{i\neq j} \eval{\frac{1}{D_j}}_{k^0=E_i} \label{eq:residue}
\end{equation}

\newcommand{\ELIPT}[2]{\eta_{#1#2}^{++}}
\newcommand{\HYPER}[2]{\eta_{#1#2}^{+-}}

We can also introduce $\ELIPT{i}{j}$ called E-Surface and $\HYPER{i}{j}$ called H-Surface to simplify the algebra.

\begin{align}
    \eval{D_j}_{k^0=E_i} &= {\qty(q^0_i+E_i-q^0_j)}^2 - \overbrace{{\qty(\va{k}-\va{q}_j)}^2+m^2}^{E_j^2}\\
    &= E_i^2+{q_i^0}^2+{q_j^0}^2-2q_i^0q_j^0+2q_i^0E_i - 2q_j^0E_i-E_j^2\\
    &= \underbrace{\left(E_i+E_j+q_i^0-q_j^0\right)}_{\ELIPT{i}{j}}\underbrace{\left(E_i-E_j+q_i^0-q_j^0\right)}_{\HYPER{i}{j}} \label{eq:surfaces}
\end{align}

\newcommand{\BLTD}[3]{2E_{#1}\ELIPT{#1}{#2}\HYPER{#1}{#2}\ELIPT{#1}{#3}\HYPER{#1}{#3}}
\begin{equation}
    I =  \int \frac{\dd[3]{k}}{{(2\pi)}^3}\qty(\frac{1}{\BLTD{1}{2}{3}}+\frac{1}{\BLTD{2}{1}{3}}+\frac{1}{\BLTD{3}{1}{2}}) \label{eq:naive_ltd}
\end{equation}

While this is a nice expression, it is not yet suited for numeric integration, since all of the H-Surfaces contain singularities which would make the procedure numerically unstable. It turns out however that these singularities are spurious and can be removed by purely algebraic manipulations, shown in detail in Appendix~\ref{sec:improved_ltd}. The resulting expression contains only E-Surfaces.

\begin{align*}
I(q,p,m_\psi)
&=  \int \frac{\dd[3]{k}}{{(2\pi)}^3} \frac{1}{(2E_1)(2E_2)(2E_3)} \Bigg[\\
  &\frac{1}{\ELIPT{2}{1}\ELIPT{3}{1}}
+ \frac{1}{\ELIPT{1}{2}\ELIPT{1}{3}}
+ \frac{1}{\ELIPT{1}{2}\ELIPT{3}{2}}
+ \frac{1}{\ELIPT{2}{1}\ELIPT{2}{3}}
+ \frac{1}{\ELIPT{1}{3}\ELIPT{2}{3}}
+ \frac{1}{\ELIPT{3}{1}\ELIPT{3}{2}}
\Bigg] \label{eq:improved_ltd}
\end{align*}